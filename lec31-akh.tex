\documentclass{lecturenotes}

\usepackage[colorlinks,urlcolor=blue]{hyperref}
\usepackage{doi}
\usepackage{xspace}
\usepackage{fontspec}
\usepackage{enumerate}
\usepackage{mathpartir}
\usepackage{pl-syntax/pl-syntax}
\usepackage{forest}
\usepackage{stmaryrd}
\usepackage{epigraph}
\usepackage{xspace}
\usepackage{bbm}
\usepackage{tikz-cd}

\setsansfont{Fira Code}

\newcommand{\abs}[2]{\ensuremath{\lambda #1.\,#2}}
\newcommand{\tabs}[3]{\ensuremath{\lambda #1 \colon #2.\,#3}}
\newcommand{\dbabs}[1]{\ensuremath{\lambda.\,#1}}
\newcommand{\dbind}[1]{\ensuremath{\text{\textasciigrave}#1}}
\newcommand{\app}[2]{\ensuremath{#1\;#2}}
\newcommand{\utype}{\textsf{unit}\xspace}
\newcommand{\unit}{\ensuremath{\textsf{(}\mkern0.5mu\textsf{)}}}
\newcommand{\prodtype}[2]{\ensuremath{#1 \times #2}}
\newcommand{\pair}[2]{\ensuremath{(#1, #2)}}
\newcommand{\projl}[1]{\ensuremath{\pi_1\mkern2mu#1}}
\newcommand{\projr}[1]{\ensuremath{\pi_2\mkern3mu#1}}
\newcommand{\sumtype}[2]{\ensuremath{#1 + #2}}
\newcommand{\injl}[1]{\ensuremath{\textsf{inj}_1\mkern2mu#1}}
\newcommand{\injr}[1]{\ensuremath{\textsf{inj}_2\mkern3mu#1}}
\newcommand{\case}[5]{\ensuremath{\textsf{case}\mkern5mu#1\mkern5mu\textsf{of}\mkern5mu\injl{#2} \Rightarrow #3;\mkern5mu\injr{#4} \Rightarrow #5\mkern5mu\textsf{end}}}
\newcommand{\vtype}{\textsf{void}\xspace}
\newcommand{\vcase}[1]{\ensuremath{\textsf{case}\mkern5mu#1\mkern5mu\textsf{of}\mkern5mu\textsf{end}}}
\newcommand{\rectype}[2]{\ensuremath{\mu #1.\,#2}}
\newcommand{\roll}[1]{\textsf{roll}\mkern2mu#1}
\newcommand{\unroll}[1]{\textsf{unroll}\mkern2mu#1}
\newcommand{\fatype}[2]{\ensuremath{\forall #1.\,#2}}
\newcommand{\Abs}[2]{\Lambda #1.\,#2}
\newcommand{\App}[2]{#1\;[#2]}
\newcommand{\extype}[2]{\ensuremath{\exists #1.\,#2}}
\newcommand{\pack}[3]{\ensuremath{\textsf{pack}\mkern5mu#1\mathrel{\textsf{as}}#2\mathrel{\textsf{in}}#3}}
\newcommand{\unpack}[4]{\ensuremath{\textsf{unpack}\mkern5mu#1\mathrel{\textsf{as}} #2, #3 \mathrel{\textsf{in}} #4}}

\newcommand{\FV}{\text{FV}}
\newcommand{\BV}{\text{BV}}

\newcommand{\toform}[1]{\ensuremath{\lceil #1 \rceil}}
\newcommand{\totype}[1]{\ensuremath{\lfloor #1 \rfloor}}

\newcommand{\neutral}[1]{#1\;\text{ne}}
\newcommand{\nf}[1]{#1\;\text{nf}}

\newcommand{\subtype}{\ensuremath{\mathrel{\mathord{<}\mathord{:}}}}

\newcommand{\pub}{\text{public}}
\newcommand{\priv}{\text{secret}}

\newcommand{\at}{\ensuremath{\mathrel{@}}}

\newcommand{\obj}[1]{\ensuremath{\mathcal{O}(#1)}}
\renewcommand{\hom}[3][]{\ensuremath{\text{Hom}_{#1}(#2, #3)}}
\newcommand{\id}[1][]{\ensuremath{\mathbbm{1}_{#1}}}

\newcommand{\Set}{\textbf{Set}\xspace}
\newcommand{\Rel}{\textbf{Rel}\xspace}
\newcommand{\Type}{\textbf{Type}\xspace}

\newcommand{\op}[1]{\ensuremath{{#1}^{\text{op}}}}

\newcommand{\prodmor}[2]{\ensuremath{\langle #1, #2 \rangle}}
\newcommand{\inl}{\text{inl}\xspace}
\newcommand{\inr}{\text{inr}\xspace}
\newcommand{\coprodmor}[2]{\ensuremath{[ #1, #2 ]}}

\title{Cartesian Closed Categories}
\coursenumber{CSE 410/510}
\coursename{Programming Language Theory}
\lecturenumber{31}
\semester{Spring 2025}
\professor{Professor Andrew K. Hirsch}

\begin{document}
\maketitle

\begin{itemize}
\item Now we have basic type constructions in categories, but we don't have any equivalent of the function type.
\item The goal today is to see that equivalent: an \emph{exponential object}.
\item Categories where you can always find an exponential object are called \emph{Cartesian-closed categories (CCCs)}.
\end{itemize}

\begin{defn}[Exponential Object]
  Let $X$ and $Y$ be objects of $\mathcal{C}$.
  Then $Z$ is called an \emph{exponential} of $X$ and $Y$ if there is a morphism $\epsilon : Z \times X \to Y$.
  Moreover, for every morphism $f : A \times X \to Y$, there is a unique morphism $\tilde{f} : A \to Z$ such that $$(\tilde{f} \times \id[X]); \epsilon = f$$
  Here, $f \times g$, where $f : A \to B$ and $g : C \to D$ is the morphism $A \times C \to B \times D$ defined by $f \times g = \prodmor{\pi_1; f}{\pi_2; g}$.
  Intuitively, it ``does $f$ on the left and $g$ on the right.''
\end{defn}

\begin{itemize}
\item We call $\epsilon$ \emph{evaluation} and $\tilde{f}$ \emph{the exponential transpose of $f$}.
\item We write $Y^X$ for \emph{the} exponential of $X$ and $Y$.
  We justify the word ``the'' with the following theorem.
\end{itemize}

\begin{thm}[Uniqueness of Exponentials]
  Exponentials are unique up to isomorphism.
\end{thm}
\begin{proof}
  Let $A$ and $B$ be two exponentials for $X$ and $Y$.
  Write the exponential transposes as such:
  \begin{mathpar}
    \infer{f : C \times X \to Y}{\tilde{f} : C \to A} \and \infer{f : C \times X \to Y}{\bar{f} : C \to B}
  \end{mathpar}
  Note that we're abusing the inference-rule notation here---this is just defining the syntax, we are \emph{not} defining inference rules.
  We write $\epsilon : A \times X \to Y$ and $\epsilon' : B \times X \to Y$.

  Note that there is a unique morphism $\tilde{\epsilon} : A \to A$ such that $(\tilde{\epsilon} \times \id[X]); \epsilon = \epsilon$.
  But certainly $\id[A]$ fits that role, so $\tilde{\epsilon} = \id[A]$.
  Similarly, $\bar{\epsilon'} = \id[B]$.

  We also have $\bar{\epsilon} : A \to B$ and $\tilde{\epsilon'} : B \to A$, which are unique such that $(\bar{\epsilon} \times \id[X]); \epsilon' = \epsilon$ and $(\tilde{\epsilon'} \times \id[X]); \epsilon = \epsilon'$.
  Note that $$((\bar{\epsilon}; \tilde{\epsilon'}) \times \id[X]); \epsilon = (\bar{\epsilon} \times \id[X]); (\tilde{\epsilon'} \times \id[X]); \epsilon = (\bar{\epsilon} \times \id[X]); \epsilon' = \epsilon$$ and $$((\tilde{\epsilon'}; \bar{\epsilon}) \times \id[X]); \epsilon' = (\tilde{\epsilon'} \times \id[X]); (\bar{\epsilon} \times \id[X]); \epsilon' = (\tilde{\epsilon'} \times \id[X]); \epsilon = \epsilon'$$
  But by uniqueness, these must be $\id[A]$ and $\id[B]$, respectively.
  Thus, $A$ and $B$ are isomorphic, as desired.
\end{proof}

\begin{defn}[Cartesian-Closed Category (CCC)]
  A category~$\mathcal{C}$ is Cartesian closed if $\mathcal{C}$ has all objects and all exponentials.
  In other words, for every pair of objects $X, Y \in \obj{\mathcal{C}}$, we have objects $X \times Y$ and $Y^X$ obeying the laws of products and exponentials, respectively.
\end{defn}

\Set is Cartesian closed: for sets $X$ and $Y$, we always have the Cartesian product $X \times Y$ as the product, and the set $Y^X = \{f \mid f : X \to Y\}$ as an exponential.
The evaluation is given by $\epsilon(f, x) = f(x)$, while for $f : A \times X \to Y$, we get $\tilde{f}(a) = \{(x, f(a, x)) \mid x \in X\}$ as a transpose.

Moreover, for STLC + (unit and products), \Type is Cartesian closed, with exponentials given by arrow types.
We get $\epsilon = x : (\tau_1 \to \tau_2) \times \tau_1 \vdash \app{(\projl{x})}{(\projr{x})} : \tau_2$, and for $f$ such that $x : A \times X \vdash f : Y$, we get $$\tilde{f} = a : A \vdash \tabs{x}{X}{f[x \mapsto (a, x)]} : X \to Y$$

Note that in STLC, $f \times g = w : A \times B \vdash \pair{f [x \mapsto \projl{w}]}{g[y \mapsto \projr{w}]} : C \times D$ where $x : A \vdash f : C$ and $y : B \vdash g : D$.
Then we get
$$
\begin{array}{ll}
  (\tilde{f} \times \id[A \to B]); \epsilon &= \epsilon [ x \mapsto \tilde{f} \times \id[A \times B]]\\
                              &= \epsilon [x \mapsto \pair{\tilde{f} [x \mapsto \projl{x}]}{\projr{x}}]\\
                              &=\app{(\projl{\pair{\tilde{f} [x \mapsto \projl{x}]}{\projr{x}}})}{(\projr{\pair{\tilde{f} [x \mapsto \projl{x}]}{\projr{x}}})}\\
                              &=\app{(\tilde{f} [x \mapsto \projl{x}])}{(\projr{x})}\\
                              &=\app{(\tabs{x}{X}{f[x \mapsto \pair{\projl{x}}{x}]})}{\projr{x}}\\
                              &\to_\beta f[x \mapsto \pair{\projl{x}}{\projr{x}}]\\
                              &\cong_\eta f[x \mapsto x]\\
                              &= f
\end{array}
$$

Moreover, let $\mathcal{C}$ be a CCC with a final object~$1$.
Then we can translate STLC + unit and products into $\mathcal{C}$.
When we do so, we translate a typing derivation $x_1 : \tau_1, \dots, x_n : \tau_n \vdash e : \tau$ to a morphism $$ \llbracket \tau_1 \rrbracket \times \dots \times \llbracket \tau_n \rrbracket \to \llbracket \tau \rrbracket$$
\begin{mathpar}
  \llbracket \utype \rrbracket = 1 \and
  \llbracket \tau_1 \times \tau_2 \rrbracket = \llbracket \tau_1 \rrbracket \times \llbracket \tau_2 \rrbracket \and
  \llbracket \tau_1 \to \tau_2 \rrbracket = {\llbracket \tau_2 \rrbracket}^{\llbracket \tau_1 \rrbracket}\\
  \left\llbracket \infer{ }{x_1 : \tau_1, \dots, x_n : \tau_n \vdash x _i : \tau_i} \right\rrbracket = \pi_i \and
  \left\llbracket \infer{ }{\Gamma \vdash \unit : \utype} \right\rrbracket = \mathord{?}_{\llbracket \Gamma \rrbracket} \\
  \left\llbracket \infer{\infer{\text{pf}}{{\Gamma, x : \tau_1 \vdash e : \tau_2}}}{\Gamma \vdash \tabs{x}{\tau_1}{e} : \tau_1 \to \tau_2}\right\rrbracket = \widetilde{\llbracket \text{pf} \rrbracket} \and
  \left\llbracket\infer{\infer{\text{pf}_1}{\Gamma \vdash f : \tau_1 \to \tau_2}\\ \infer{\text{pf}_2}{\Gamma \vdash a : \tau_1}}{\Gamma \vdash \app{f}{a} : \tau_2} \right\rrbracket = \langle\llbracket \text{pf}_1 \rrbracket, \llbracket \text{pf}_2 \rrbracket\rangle; \epsilon\\
  \left\llbracket\infer{\infer{\text{pf}_1}{\Gamma \vdash e_1 : \tau_1}\\ \infer{\text{pf}_2}{\Gamma \vdash e_2 : \tau_2}}{\Gamma \vdash \pair{e_1}{e_2} : \prodtype{\tau_1}{\tau_2}} \right\rrbracket = \prodmor{\llbracket \text{pf}_1\rrbracket}{\llbracket\text{pf}_2\rrbracket} \and
  \left\llbracket\infer{\infer{\text{pf}}{\Gamma \vdash e : \tau_1 \times \tau_2}}{\Gamma \vdash \projl{e} : \tau_1}\right\rrbracket = \llbracket\text{pf}\rrbracket; \pi_1 \and
  \left\llbracket\infer{\infer{\text{pf}}{\Gamma \vdash e : \tau_1 \times \tau_2}}{\Gamma \vdash \projr{e} : \tau_1}\right\rrbracket = \llbracket\text{pf}\rrbracket; \pi_2
\end{mathpar}

Moreover, by adding base types, we can make it so that we can build a model of any CCC in STLC + products.
Thus, we can reason about CCCs by reasoning about STLC.
One fact that we will not use today, but which is very useful for doing category theory, is that both anything that can be proven $\alpha, \beta, \eta$ equivalent in STLC + products is equal in CCCs, and that anything that can be proven equal in all CCCs can be proven equal in STLC + products.
Thus, we say that STLC + products is the \emph{internal language of CCCs}.
This is a super useful fact about categories, as it turns large algebraic proofs into simple $\lambda$~calculus calculations.



\end{document}

%%% Local Variables:
%%% mode: latex
%%% TeX-master: t
%%% TeX-engine: luatex
%%% End:
