\documentclass{lecturenotes}

\usepackage[colorlinks,urlcolor=blue]{hyperref}
\usepackage{doi}
\usepackage{xspace}
\usepackage{fontspec}
\usepackage{enumerate}
\usepackage{mathpartir}
\usepackage{pl-syntax/pl-syntax}
\usepackage{forest}
\usepackage{stmaryrd}
\usepackage{epigraph}
\usepackage{xspace}
\usepackage{bbm}
\usepackage{tikz-cd}
\usepackage{unicode-math}
\usepackage{agda}

\usepackage{newunicodechar}
\newunicodechar{₁}{\ensuremath{{}_{1}}}
\newunicodechar{₂}{\ensuremath{{}_{2}}}

\setsansfont{Fira Code}
\setmathfont{AsanaMath}

\newcommand{\abs}[2]{\ensuremath{\lambda #1.\,#2}}
\newcommand{\tabs}[3]{\ensuremath{\lambda #1 \colon #2.\,#3}}
\newcommand{\dbabs}[1]{\ensuremath{\lambda.\,#1}}
\newcommand{\dbind}[1]{\ensuremath{\text{\textasciigrave}#1}}
\newcommand{\app}[2]{\ensuremath{#1\;#2}}
\newcommand{\utype}{\textsf{unit}\xspace}
\newcommand{\unit}{\ensuremath{\textsf{(}\mkern0.5mu\textsf{)}}}
\newcommand{\prodtype}[2]{\ensuremath{#1 \times #2}}
\newcommand{\pair}[2]{\ensuremath{(#1, #2)}}
\newcommand{\projl}[1]{\ensuremath{\pi_1\mkern2mu#1}}
\newcommand{\projr}[1]{\ensuremath{\pi_2\mkern3mu#1}}
\newcommand{\sumtype}[2]{\ensuremath{#1 + #2}}
\newcommand{\injl}[1]{\ensuremath{\textsf{inj}_1\mkern2mu#1}}
\newcommand{\injr}[1]{\ensuremath{\textsf{inj}_2\mkern3mu#1}}
\newcommand{\case}[5]{\ensuremath{\textsf{case}\mkern5mu#1\mkern5mu\textsf{of}\mkern5mu\injl{#2} \Rightarrow #3;\mkern5mu\injr{#4} \Rightarrow #5\mkern5mu\textsf{end}}}
\newcommand{\vtype}{\textsf{void}\xspace}
\newcommand{\vcase}[1]{\ensuremath{\textsf{case}\mkern5mu#1\mkern5mu\textsf{of}\mkern5mu\textsf{end}}}
\newcommand{\rectype}[2]{\ensuremath{\mu #1.\,#2}}
\newcommand{\roll}[1]{\textsf{roll}\mkern2mu#1}
\newcommand{\unroll}[1]{\textsf{unroll}\mkern2mu#1}
\newcommand{\fatype}[2]{\ensuremath{\forall #1.\,#2}}
\newcommand{\Abs}[2]{\Lambda #1.\,#2}
\newcommand{\App}[2]{#1\;[#2]}
\newcommand{\extype}[2]{\ensuremath{\exists #1.\,#2}}
\newcommand{\pack}[3]{\ensuremath{\textsf{pack}\mkern5mu#1\mathrel{\textsf{as}}#2\mathrel{\textsf{in}}#3}}
\newcommand{\unpack}[4]{\ensuremath{\textsf{unpack}\mkern5mu#1\mathrel{\textsf{as}} #2, #3 \mathrel{\textsf{in}} #4}}

\newcommand{\FV}{\text{FV}}
\newcommand{\BV}{\text{BV}}

\newcommand{\toform}[1]{\ensuremath{\lceil #1 \rceil}}
\newcommand{\totype}[1]{\ensuremath{\lfloor #1 \rfloor}}

\newcommand{\neutral}[1]{#1\;\text{ne}}
\newcommand{\nf}[1]{#1\;\text{nf}}

\newcommand{\subtype}{\ensuremath{\mathrel{\mathord{<}\mathord{:}}}}

\newcommand{\pub}{\text{public}}
\newcommand{\priv}{\text{secret}}

\newcommand{\at}{\ensuremath{\mathrel{@}}}

\newcommand{\obj}[1]{\ensuremath{\mathcal{O}(#1)}}
\renewcommand{\hom}[3][]{\ensuremath{\text{Hom}_{#1}(#2, #3)}}
\newcommand{\id}[1][]{\ensuremath{\mathbbm{1}_{#1}}}

\newcommand{\Set}{\textbf{Set}\xspace}
\newcommand{\Rel}{\textbf{Rel}\xspace}
\newcommand{\Type}{\textbf{Type}\xspace}
\renewcommand{\Cat}{\textbf{Cat}\xspace}

\newcommand{\op}[1]{\ensuremath{{#1}^{\text{op}}}}

\newcommand{\prodmor}[2]{\ensuremath{\langle #1, #2 \rangle}}
\newcommand{\inl}{\text{inl}\xspace}
\newcommand{\inr}{\text{inr}\xspace}
\newcommand{\coprodmor}[2]{\ensuremath{[ #1, #2 ]}}

\title{Functors and Transformations}
\coursenumber{CSE 410/510}
\coursename{Programming Language Theory}
\lecturenumber{32}
\semester{Spring 2025}
\professor{Professor Andrew K. Hirsch}

\begin{document}
\maketitle

\begin{itemize}
\item We've now explored categories and some of the structure they can have.
\item Importantly, we can think of categories as little programming languages, and structure in those categories as the existence of certain types and programs.
\item One lesson of category theory is that the relationships between objects of study is just as important (perhaps more so!) than the object itself.
  \begin{itemize}
  \item After all, the objects of a category are just black boxes.
  \item All of the structure is in the morphisms.
  \end{itemize}
\item This shows up in programming languages as well: often, the best way to understand a language is by translating it into another language.
\item Today, we'll see how to frame these translations in categories.
\end{itemize}

\section{Functors}
\label{sec:functors}

A transformation of categories is called a \emph{functor}.

\begin{defn}[Functor]
  A \emph{functor}~$F$ from a category~$\mathcal{C}$ to another category~$\mathcal{D}$ (written $F : \mathcal{C} \to \mathcal{D}$) consists of:
  \begin{itemize}
  \item A mapping from objects of $\mathcal{C}$ to objects of $\mathcal{D}$.
    For $X \in \obj{\mathcal{C}}$, we write $F\,X \in \obj{\mathcal{D}}$.
  \item For every pair of objects $X, Y \in \obj{\mathcal{C}}$, a mapping from $\hom[\mathcal{C}]{X}{Y}$ to $\hom[\mathcal{D}]{F\,X}{F\,Y}$.
    If $f : X \to Y$, we write $F\,f : F\,X \to F\,Y$.
  \end{itemize}
  obeying the following laws:
  \begin{itemize}
  \item $F$ preserves the identity: $F\,\id[X] = \id[F\,X]$
  \item $F$ preserves composition: $F\,(f; g) = (F\,f); (F\,g)$
  \end{itemize}
\end{defn}

\begin{itemize}
\item If we think about categories as programming languages (where the objects are types and the morphisms are programs), then we can think of this as a translation from one typed programming language to another.
\item Imagine that we have two such categories, $\mathcal{A}$ and $\mathcal{B}$.
\item Then a functor $F : \mathcal{A} \to \mathcal{B}$ consists of:
  \begin{itemize}
  \item A translation of types: every type in $\mathcal{A}$ is translated to some type in $\mathcal{B}$ that ``can represent the same thing''
  \item A translation of programs: every well-typed program in $\mathcal{A}$ is translated to a well-typed program in $\mathcal{B}$.
  \end{itemize}
\item A functor $F : \mathcal{C} \to \mathcal{C}$ is often called an \emph{endofunctor} on $\mathcal{C}$.
\item Some examples:
  \begin{itemize}
  \item For every category $\mathcal{C}$, there is a functor $\id[\mathcal{C}] : \mathcal{C} \to \mathcal{C}$ that takes every object $X \in \obj{\mathcal{C}}$ to itself and every morphism $f : X \to Y$ to itself.
    This is called the \emph{identity functor} on $\mathcal{C}$.
  \item As an example: there is a functor $U : \Set \to \Rel$ which takes a set $X$ to itself, and which takes a function $f : X \to Y$ to itself (now viewed as a relation $f \subseteq X \times Y$ that just happens to be functional).
    \begin{itemize}
    \item Such a functor, which acts like the identity functor but is between two different categories, is called an \emph{identity-carried} functor.
    \end{itemize}
    \newpage
  \item There is another functor $U : \Rel \to \Set$ which takes a set $X$ to its powerset $\mathcal{P}(X)$ and a relation $R \subseteq X \times Y$ to $f(A) = \bigcup_{x \in A} \{y \in Y \mid x \mathrel{R} y\}$.
  \item There is an endofunctor $\textsf{list} : \Type \to \Type$ that takes a type $\tau$ to the type $\textsf{list}\,\tau$ and a program $p : \tau \to \sigma$ to the program $\textsf{map}\,p : \textsf{list}\,\tau \to \textsf{list}\,\sigma$.
    A similar endofunctor exists for any data structure that can define a (well-behaved) \textsf{map} function, including:
    \begin{itemize}
    \item \textsf{tree}, the endofunctor of binary trees
    \item \textsf{set}, the endofunctor of (abstract) sets
    \item $\textsf{map}\,k$, the endofunctor of $k$-keyed maps
    \end{itemize}
  \end{itemize}
\item Imagine that we have categories~$\mathcal{C}$, $\mathcal{D}$, and~$\mathcal{E}$, with functors~$F : \mathcal{C} \to \mathcal{D}$ and~$G : \mathcal{D} \to \mathcal{E}$.
  Then we can define $F; G : \mathcal{C} \to \mathcal{E}$ as the functor that takes an object $X \in \obj{\mathcal{C}}$ to $G \, (F\, X)$ and a morphism $f : X \to Y$ to $G\,(F\,f)$.
  Note that since $F \, f : F\,X \to F\,Y$, we get $G\, (F\,f) : G\,(F\,X) \to G\,(F\,Y)$ as required.
\item This means that there is a category \Cat, where objects are categories and morphisms are functors!
\end{itemize}

\section{Natural Transformation}
\label{sec:natur-transf}

\begin{itemize}
\item In the same way that functors allow us to study categories by looking at how they transform, we sometimes want to study functors by understanding how they transform.
\item We can do that by studying \emph{natural transformations} between functors.
\end{itemize}

\begin{defn}[Natural Transformation]
  If $F : \mathcal{C} \to \mathcal{D}$ and $G : \mathcal{C} \to \mathcal{D}$ are both functors, then a \emph{natural transformation} from $F$ to $G$ (written $\alpha : F \Rightarrow G$) consists of a morphism $\alpha_X \in \hom[\mathcal{D}]{F\,X}{G\,X}$ for every object $X \in \obj{\mathcal{C}}$ such that the following diagram commutes for every $f : X \to Y$ in $\mathcal{C}$:
  \begin{center}
    \begin{tikzcd}
      F\,X \ar[r, "F\,f"] \ar[d, "\alpha_X"'] & F\,Y \ar[d, "\alpha_Y"]\\
      G\,X \ar[r,"G\,f"'] & G\,Y
    \end{tikzcd}
  \end{center}
\end{defn}

\begin{itemize}
\item If we think of $F$ and $G$ as data structures represented by endofunctors, then we can think of $\alpha : F \Rightarrow G$ as a polymorphic function.
  \begin{itemize}
  \item For instance, there is a natural transformation $\textsf{flatten} : \textsf{tree} \Rightarrow \textsf{list}$ defined by the function
    \begin{code}[hide]
      import Relation.Binary.PropositionalEquality as Eq
      open Eq using (_≡_; _≢_; refl; cong; sym)
      open Eq.≡-Reasoning

      data list (X : Set) : Set where
        [] : list X
        _∷_ : X → list X → list X

      data tree (X : Set) : Set where
        leaf : tree X
        node : tree X → X → tree X → tree X

      _++_ : {X : Set}  → list X → list X → list X
      [] ++ ys = ys
      (x ∷ xs) ++ ys = x ∷ (xs ++ ys)

      map : {X Y : Set} → (f : X → Y) → list X → list Y
      map f [] = []
      map f (x ∷ xs) = f x ∷ map f xs

      treemap : {X Y : Set} → (f : X → Y) → tree X → tree Y
      treemap f leaf = leaf
      treemap f (node t₁ x t₂) = node (treemap f t₁) (f x) (treemap f t₂)

      map-append : {X Y : Set} → (f : X → Y) → (xs ys : list X) → map f (xs ++ ys) ≡ map f xs ++ map f ys
      map-append f [] ys = refl
      map-append f (x ∷ xs) ys = cong (f x ∷_) (map-append f xs ys)
    \end{code}
    \begin{code}
      flatten : {X : Set} → tree X → list X
      flatten leaf = []
      flatten (node t₁ x t₂) = (flatten t₁)  ++ (x ∷ flatten t₂)
    \end{code}
    \newpage
  \item The naturality condition says that \textsf{treemap} and \textsf{map} play nicely with \textsf{flatten}, which we can easily prove:
    \begin{code}
      flatten-map : {X Y : Set} → (f : X → Y) → (t : tree X) →
                              map f (flatten t) ≡ flatten (treemap f t)
      flatten-map f leaf = refl
      flatten-map f (node t₁ x t₂) =
        begin
          map f (flatten t₁ ++ (x ∷ flatten t₂))
        ≡⟨ map-append f (flatten t₁) (x ∷ flatten t₂) ⟩
          map f (flatten t₁) ++ (f x ∷ map f (flatten t₂))
        ≡⟨ cong (_++ (f x ∷ map f (flatten t₂))) (flatten-map f t₁) ⟩
          flatten (treemap f t₁) ++ (f x ∷ map f (flatten t₂))
        ≡⟨ cong (flatten (treemap f t₁) ++_) (cong (f x ∷_) (flatten-map f t₂)) ⟩
          flatten (treemap f t₁) ++ (f x ∷ flatten (treemap f t₂))
        ≡⟨⟩
          flatten (node (treemap f t₁) (f x) (treemap f t₂))
        ≡⟨⟩
          flatten (treemap f (node t₁ x t₂))
        ∎
    \end{code}
  \item In fact, this can be proven as a free theorem of any program of type $\fatype{X}{\textsf{tree}\,X \to \textsf{list}\,X}$!
  \end{itemize}
\item For any functor $F : \mathcal{C} \to \mathcal{D}$, there is a natural transformation $\id[F] : F \Rightarrow F$ with components $(\id[F])_X = \id[F\,X]$.
\item Moreover, for any pair of natural transformations $\alpha : F \Rightarrow G$ and $\beta : G \Rightarrow H$, there is a natural transformation $\alpha; \beta : F \Rightarrow H$ with components $(\alpha; \beta)_X = \alpha_X; \beta_X$.
  The naturality condition can be seen via a pasting argument:
  \begin{center}
    \begin{tikzcd}
      F\,X \ar[d,"F\,f"'] \ar[r,"\alpha_X"] & G\,X\ar[d,"G\,f"] \ar[r,"\beta_X"] & H\,X\ar[d,"H\,f"]\\
      F\,Y \ar[r,"\alpha_Y"'] & G\,Y\ar[r,"\beta_Y"'] & H\,Y
    \end{tikzcd}
  \end{center}
  Since the two rectangles each commute, the outer rectangle must commute.
  This style of proof is sometimes called a ``pasting argument'' (especially when things are this simple, and sometimes called a ``diagram chase'' (especially in complex cases).
\item Together, this means that for any pair of categories~$\mathcal{C}$ and~$\mathcal{D}$, there is a \emph{functor category}~$[\mathcal{C}, \mathcal{D}]$ where objects are functors $\mathcal{C} \to \mathcal{D}$ and morphisms are natural transformations.
  \begin{itemize}
  \item For those interested, this makes \Cat a \emph{2-category}, which is just a category where the morphisms between any two objects form a category.
  \item There is a notion of $n$-category for any $n$ including $\infty$.
    This notion is really important for the study of homotopy type theory, among others.
  \item There is another related notion of a \emph{enriched} category.
    An enriched category~$\mathcal{C}$ over another category~$\mathcal{D}$ has $\hom[\mathcal{C}]{X}{Y} \in \obj{\mathcal{D}}$.
    Then, a $2$-category is just a category enriched over \Cat.
  \end{itemize}
\end{itemize}

\end{document}

%%% Local Variables:
%%% mode: latex
%%% TeX-master: t
%%% TeX-engine: luatex
%%% TeX-command-default: "Make"
%%% End:
