\documentclass{lecturenotes}

\usepackage[colorlinks,urlcolor=blue]{hyperref}
\usepackage{doi}
\usepackage{xspace}
\usepackage{fontspec}
\usepackage{enumerate}
\usepackage{mathpartir}
\usepackage{pl-syntax/pl-syntax}
\usepackage{forest}
\usepackage{stmaryrd}
\usepackage{epigraph}
\usepackage{xspace}
\usepackage{bbm}
\usepackage{tikz-cd}
\usepackage{unicode-math}
\usepackage{agda}

\usepackage{newunicodechar}
\newunicodechar{₁}{\ensuremath{{}_{1}}}
\newunicodechar{₂}{\ensuremath{{}_{2}}}
\newunicodechar{∷}{\ensuremath{\mathrel{::}}}
\newunicodechar{∎}{\ensuremath{\textsf{\blacksquare}}}

\setsansfont{Fira Code}
\setmathfont{AsanaMath}

\newcommand{\abs}[2]{\ensuremath{\lambda #1.\,#2}}
\newcommand{\tabs}[3]{\ensuremath{\lambda #1 \colon #2.\,#3}}
\newcommand{\dbabs}[1]{\ensuremath{\lambda.\,#1}}
\newcommand{\dbind}[1]{\ensuremath{\text{\textasciigrave}#1}}
\newcommand{\app}[2]{\ensuremath{#1\;#2}}
\newcommand{\utype}{\textsf{unit}\xspace}
\newcommand{\unit}{\ensuremath{\textsf{(}\mkern0.5mu\textsf{)}}}
\newcommand{\prodtype}[2]{\ensuremath{#1 \times #2}}
\newcommand{\pair}[2]{\ensuremath{(#1, #2)}}
\newcommand{\projl}[1]{\ensuremath{\pi_1\mkern2mu#1}}
\newcommand{\projr}[1]{\ensuremath{\pi_2\mkern3mu#1}}
\newcommand{\sumtype}[2]{\ensuremath{#1 + #2}}
\newcommand{\injl}[1]{\ensuremath{\textsf{inj}_1\mkern2mu#1}}
\newcommand{\injr}[1]{\ensuremath{\textsf{inj}_2\mkern3mu#1}}
\newcommand{\case}[5]{\ensuremath{\textsf{case}\mkern5mu#1\mkern5mu\textsf{of}\mkern5mu\injl{#2} \Rightarrow #3;\mkern5mu\injr{#4} \Rightarrow #5\mkern5mu\textsf{end}}}
\newcommand{\vtype}{\textsf{void}\xspace}
\newcommand{\vcase}[1]{\ensuremath{\textsf{case}\mkern5mu#1\mkern5mu\textsf{of}\mkern5mu\textsf{end}}}
\newcommand{\rectype}[2]{\ensuremath{\mu #1.\,#2}}
\newcommand{\roll}[1]{\textsf{roll}\mkern2mu#1}
\newcommand{\unroll}[1]{\textsf{unroll}\mkern2mu#1}
\newcommand{\fatype}[2]{\ensuremath{\forall #1.\,#2}}
\newcommand{\Abs}[2]{\Lambda #1.\,#2}
\newcommand{\App}[2]{#1\;[#2]}
\newcommand{\extype}[2]{\ensuremath{\exists #1.\,#2}}
\newcommand{\pack}[3]{\ensuremath{\textsf{pack}\mkern5mu#1\mathrel{\textsf{as}}#2\mathrel{\textsf{in}}#3}}
\newcommand{\unpack}[4]{\ensuremath{\textsf{unpack}\mkern5mu#1\mathrel{\textsf{as}} #2, #3 \mathrel{\textsf{in}} #4}}

\newcommand{\FV}{\text{FV}}
\newcommand{\BV}{\text{BV}}

\newcommand{\toform}[1]{\ensuremath{\lceil #1 \rceil}}
\newcommand{\totype}[1]{\ensuremath{\lfloor #1 \rfloor}}

\newcommand{\neutral}[1]{#1\;\text{ne}}
\newcommand{\nf}[1]{#1\;\text{nf}}

\newcommand{\subtype}{\ensuremath{\mathrel{\mathord{<}\mathord{:}}}}

\newcommand{\pub}{\text{public}}
\newcommand{\priv}{\text{secret}}

\newcommand{\at}{\ensuremath{\mathrel{@}}}

\newcommand{\obj}[1]{\ensuremath{\mathcal{O}(#1)}}
\renewcommand{\hom}[3][]{\ensuremath{\text{Hom}_{#1}(#2, #3)}}
\newcommand{\id}[1][]{\ensuremath{\mathbbm{1}_{#1}}}

\newcommand{\Set}{\textbf{Set}\xspace}
\newcommand{\Rel}{\textbf{Rel}\xspace}
\newcommand{\Type}{\textbf{Type}\xspace}
\renewcommand{\Cat}{\textbf{Cat}\xspace}

\newcommand{\op}[1]{\ensuremath{{#1}^{\text{op}}}}

\newcommand{\prodmor}[2]{\ensuremath{\langle #1, #2 \rangle}}
\newcommand{\inl}{\text{inl}\xspace}
\newcommand{\inr}{\text{inr}\xspace}
\newcommand{\coprodmor}[2]{\ensuremath{[ #1, #2 ]}}

\newtheorem{ex}{Example}

\title{Functors and Transformations}
\coursenumber{CSE 410/510}
\coursename{Programming Language Theory}
\lecturenumber{32}
\semester{Spring 2025}
\professor{Professor Andrew K. Hirsch}

\begin{document}
\maketitle

We started lecture by making the observation that there is a tight connection between programming languages and category theory.
Of course, as we learned earlier, programming languages can be modeled as categories.
Additionally, in the previous lecture we learned about Closed Cartesian Categories (CCC), which themselves can be thought of as programming languages.

We also made the observation that, when considering a category, we are not really interested in the objects in that category.
Rather, we are more interested in the morphisms, or the transformations between objects.
The morphisms are what give structure to a category.

In this lecture, we add more layers of abstraction to categories in the form of \emph{functors}, which are transformations between categories, and \emph{natural transformations}, which are transformations between functors.

\section{Functors}

Informally, a \emph{functor} is a structure-preserving mapping between categories.
That is, a functor maps one category to another so that the connections between objects are preserved.

\begin{defn}[Functor]
  A functor $F : \mathcal{C} \to \mathcal{D}$ consists of:

  \begin{itemize}
    \item A transformation between objects $X \mapsto \app{F}{X}$.
    \item A transformation between morphisms $f : A \to B \mapsto \app{F}{f} : \app{F}{A} \to \app{F}{X}$,
    
    \;where $f$ is in $\hom[\mathcal{C}]{A}{B}$ and $\app{F}{f}$ is in $\hom[\mathcal{D}]{\app{F}{A}}{\app{F}{B}}$.

    \item Subject to:
    \begin{itemize}
      \item $\app{F}{(\id[X])} = \id[\app{F}{X}]$
      \item $\app{F}{(f ; g)} = \app{F}{f}; \app{F}{g}$
    \end{itemize}
  \end{itemize}
\end{defn}

In the context of programming languages, we can define a functor $F$ to convert types in language $\mathcal{C}$ to types in language $\mathcal{D}$.
Or, $F$ can be a transformation that converts programs from $\mathcal{C}$ to $\mathcal{D}$.

We call a functor that maps a category to itself, $F : \mathcal{C} \to \mathcal{C}$, an \emph{endofunctor}.\\

Let us consider some examples.

\begin{ex}
  For every $\mathcal{C}$, there is a functor $\id[\mathcal{C}] : \mathcal{C} \to \mathcal{C}$, where

  \begin{mathpar}
    \begin{array}{ccc}
      \id[\mathcal{C}](X) = X && \app{\id[\mathcal{C}]}{f} = f
    \end{array}
  \end{mathpar}
\end{ex}

\begin{ex}
  Let $F : \mathcal{C} \to \mathcal{D}$ and $G : \mathcal{D} \to \mathcal{E}$. Then,

  \begin{mathpar}
    \begin{array}{ccc}
      F ; G : \mathcal{C} \to \mathcal{E} && \app{(F ; G)}{X} = \app{G}{(\app{F}{X})} \\
      && \app{(F ; G)}{f} = \app{G}{(\app{F}{f})}
    \end{array}
  \end{mathpar}
\end{ex}

\begin{ex}
  There is a category \Cat of all categories where the morphisms are functors.
\end{ex}

\begin{ex}
  Consider the \textsf{list} type in Agda: $\textsf{list : Type} \to \textsf{Type}$.\\

  The type \textsf{list X} is the type of \textsf{list}s where every element has type \textsf{X}.\\

  If \textsf{f} is a function, then \textsf{list f} corresponds to \textsf{map f}, where \textsf{map} is the \textsf{map} function for \textsf{list}s.
\end{ex}

\section{Natural Transformations}

\begin{defn}[Natural Transformation]
  If $F, G : \mathcal{C} \to \mathcal{D}$, then a \emph{natural transformation} $\alpha : F \Rightarrow G$ consists of a collection of morphisms $\alpha_X : \app{F}{X} \to \app{G}{X}$, where the following diagram must commute for every $f : X \to Y$ in $\mathcal{C}$:

  \begin{center}
    \begin{tikzcd}
      F\,X \ar[r, "F\,f"] \ar[d, "\alpha_X"'] & F\,Y \ar[d, "\alpha_Y"]\\
      G\,X \ar[r,"G\,f"'] & G\,Y
    \end{tikzcd}
  \end{center}
\end{defn}

Let us consider some examples.

\begin{ex}
  Consider the \textsf{list} and \textsf{tree} types.\\

  If we can define a function $\alpha$ that translates \textsf{tree X} to \textsf{list X}, for any \textsf{X}, then $\alpha$ is a natural transformation from \textsf{tree} to \textsf{list}.
  
  \begin{center}
    \begin{tikzcd}
      \textsf{tree X} \ar[r, "\textsf{tree f}"] \ar[d, "\alpha_X"'] & \textsf{tree Y} \ar[d, "\alpha_Y"]\\
      \textsf{list X} \ar[r,"\textsf{list f}"'] & \textsf{list Y}
    \end{tikzcd}
  \end{center}
\end{ex}

The following is a free theorem: any polymorphic function from \textsf{tree} to \textsf{list} is a natural transformation.
In fact, this is true for any two functors defined as variants.

\begin{ex}
  Let $F : \mathcal{C} \to \mathcal{D}$.
  Then:

  \begin{mathpar}
    \begin{array}{ccc}
      \id[F] : F \Rightarrow F && (\id[F])_X = \id[FX]
    \end{array}
  \end{mathpar}

  \begin{center}
    \begin{tikzcd}
      F\,X \ar[r, "F\,f"] \ar[d, "\mathbbm{1}_{FX}"'] & F\,Y \ar[d, "\mathbbm{1}_{FY}"]\\
      F\,X \ar[r,"F\,f"'] & F\,Y
    \end{tikzcd}
  \end{center}
\end{ex}

\begin{ex}
  Let $F : \mathcal{C} \to \mathcal{D}$, $G : \mathcal{C} \to \mathcal{D}$, $H : \mathcal{C} \to \mathcal{D}$, and

  \begin{mathpar}
    \begin{array}{ccc}
      \alpha : F \Rightarrow G && \beta : G \Rightarrow H
    \end{array}
  \end{mathpar}

  We can represent this pictorially with the following diagram:

  \begin{center}
    \begin{tikzcd}
      F\,X \ar[r, "\alpha_X"] \ar[d, "F\,f"'] & G\,X \ar[r, "\beta_X"] \ar[d, "G\,f"] & H\,X \ar[d, "H\,f"]\\
      F\,Y \ar[r,"\alpha_Y"'] & G\,Y \ar[r, "\beta_Y"] & H\,Y
    \end{tikzcd}
  \end{center}

  We also know that:

  \begin{mathpar}
    \begin{array}{ccc}
      \alpha ; \beta : F \Rightarrow H && (\alpha ; \beta)_X = \alpha_X ; \beta_X
    \end{array}
  \end{mathpar}

  So we can update our diagram to:

  \begin{center}
    \begin{tikzcd}
      F\,X \ar[r, "\alpha_X"] \ar[rr, bend left, "(\alpha ; \beta)_X"] \ar[d, "F\,f"'] & G\,X \ar[r, "\beta_X"] \ar[d, "G\,f"] & H\,X \ar[d, "H\,f"]\\
      F\,Y \ar[r,"\alpha_Y"'] \ar[rr, bend right, "(\alpha ; \beta)_Y"] & G\,Y \ar[r, "\beta_Y"] & H\,Y
    \end{tikzcd}
  \end{center}

  This entire diagram commutes by the \emph{pasting argument}.
  Informally, the pasting argument says that putting together commuting diagrams results in a larger commuting diagram.
\end{ex}

\end{document}