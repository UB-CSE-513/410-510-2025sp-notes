\documentclass{lecturenotes}

\usepackage[colorlinks,urlcolor=blue]{hyperref}
\usepackage{doi}
\usepackage{xspace}
\usepackage{fontspec}
\usepackage{enumerate}
\usepackage{mathpartir}
\usepackage{pl-syntax/pl-syntax}
\usepackage{forest}
\usepackage{stmaryrd}
\usepackage{epigraph}
\usepackage{xspace}

\setsansfont{Fira Code}

\newcommand{\abs}[2]{\ensuremath{\lambda #1.\,#2}}
\newcommand{\tabs}[3]{\ensuremath{\lambda #1 \colon #2.\,#3}}
\newcommand{\dbabs}[1]{\ensuremath{\lambda.\,#1}}
\newcommand{\dbind}[1]{\ensuremath{\text{\textasciigrave}#1}}
\newcommand{\app}[2]{\ensuremath{#1\;#2}}
\newcommand{\utype}{\textsf{unit}\xspace}
\newcommand{\unit}{\ensuremath{\textsf{(}\mkern0.5mu\textsf{)}}}
\newcommand{\prodtype}[2]{\ensuremath{#1 \times #2}}
\newcommand{\pair}[2]{\ensuremath{(#1, #2)}}
\newcommand{\projl}[1]{\ensuremath{\pi_1\mkern2mu#1}}
\newcommand{\projr}[1]{\ensuremath{\pi_2\mkern3mu#1}}
\newcommand{\sumtype}[2]{\ensuremath{#1 + #2}}
\newcommand{\injl}[1]{\ensuremath{\textsf{inj}_1\mkern2mu#1}}
\newcommand{\injr}[1]{\ensuremath{\textsf{inj}_2\mkern3mu#1}}
\newcommand{\case}[5]{\ensuremath{\textsf{case}\mkern5mu#1\mkern5mu\textsf{of}\mkern5mu\injl{#2} \Rightarrow #3;\mkern5mu\injr{#4} \Rightarrow #5\mkern5mu\textsf{end}}}
\newcommand{\vtype}{\textsf{void}\xspace}
\newcommand{\vcase}[1]{\ensuremath{\textsf{case}\mkern5mu#1\mkern5mu\textsf{of}\mkern5mu\textsf{end}}}

\newcommand{\FV}{\text{FV}}
\newcommand{\BV}{\text{BV}}

\title{Basic Type Constructions}
\coursenumber{CSE 410/510}
\coursename{Programming Language Theory}
\lecturenumber{16}
\semester{Spring 2025}
\professor{Professor Andrew K. Hirsch}

\begin{document}
\maketitle

\begin{itemize}
\item Untyped $\lambda$~calculus is Turing-complete: you can write any program in it.
\item However, STLC is very constraining.
  In fact, it is \emph{not} Turing-complete, as we will prove later.
  (Intuitively, the type system won't let you write something recursive---to do so, you would have to have something that is both 
\item Our goal for the next few weeks is to add features to STLC in order to
  \begin{enumerate}[(1)]
  \item make it more pleasant to program in, and
  \item make it more expressive, while
  \item maintaining the safety (or at least most of the safety) that we achieved by adding types.
  \end{enumerate}
\item Today, we look at four basic type constructs: \utype, products, sum types (or coproducts), and \textsf{void}.
\end{itemize}

\section{The \utype Type}
\label{sec:unit-type}

\begin{itemize}
\item One of the most-annoying points in the last lecture was the addition of the base type.
\item After all, it was never able to type any (closed) terms, so it was kind of meaningless.
\item Let's replace it with a type that's more worthwhile: the \utype type, which has exactly one element, $\unit$.
\end{itemize}

\begin{syntax}
  \category[Types]{\tau} \alternative{\ldots} \alternative{\utype}
  \category[Expressions]{e} \alternative{\ldots} \alternative{\unit}
  \category[Values]{v} \alternative{\abs{x}{e}} \alternative{\unit}
\end{syntax}

\begin{mathpar}
  \infer*[left=unit]{ }{\Gamma \vdash \unit \colon \utype}
\end{mathpar}

\begin{itemize}
\item There is only one additional rule, the \textsc{unit} rule.
\item This rule says that $\unit$ has type \utype.
\item Since $\unit$ is a closed expression, the \utype~type is actually meaningful.
\item Moreover, $\unit$ is a value.
\item We prove one useful theorem for CBV $\lambda$~calculus with units:
\end{itemize}

\begin{thm}[\utype Inversion]
  If $\Gamma \vdash v \colon \utype$ and $v$ is a value, then $v = \unit$.
\end{thm}
\begin{proof}
  By case analysis on the proof of $\Gamma \vdash v : \utype$.

  \noindent\textbf{Case \textsc{var}:} In this case, $v$ is not a value.
  
  \noindent\textbf{Case \textsc{abs}:} In this case, $v$ has an arrow type and not type \utype.

  \noindent\textbf{Case \textsc{app}:} In this case, $v$ is not a value.

  \noindent\textbf{Case \textsc{unit}:} In this case, $v = \unit$ as required.
\end{proof}

\section{Product Types}
\label{sec:product-types}

\begin{itemize}
\item Products are used in OCaml and Agda to write tuples.
\item We can add them to STLC to get the same result.
\item For any two types~$\tau_1$ and~$\tau_2$, we need to add a product type $\prodtype{\tau_1}{\tau_2}$ which has tuples as values.
\item Of course, in order to do this we also need to add tuples to the language!
\item If that's all we add, we have a problem.
\item \textbf{Stop and Think:} What is this problem, and what do we need to add to solve it?
\item Answer: right now, if we ever put data in a tuple it gets ``trapped'' there and we can never get it out!
\item In order to fix it, we add two projections, $\projl{}$ and $\projr{}$, which allow us to get the first and second element of a tuple, respectively.
\end{itemize}

\begin{syntax}
  \category[Types]{\tau} \alternative{\ldots} \alternative{\prodtype{\tau_1}{\tau_2}}
  \category[Expressions]{e} \alternative{\ldots} \alternative{\pair{e_1}{e_2}} \alternative{\projl{e}} \alternative{\projr{e}}
  \category[Values]{v} \alternative{\ldots} \alternative{\pair{v_1}{v_2}}
  \category[Evaluation Contexts]{C} \alternative{\ldots} \alternative{\pair{C}{e}} \alternative{\pair{v}{C}} \alternative{\projl{C}} \alternative{\projr{C}}
\end{syntax}

\begin{mathpar}
  \infer*[left=projLeft]{ }{\projl{\pair{v_1}{v_2}} \to_{\text{CBV}} v_1}\and
  \infer*[right=projRight]{ }{\projr{\pair{v_1}{v_2}} \to_{\text{CBV}} v_2}
\end{mathpar}

\begin{mathpar}
  \infer*[left=pair]{\Gamma \vdash e_1 \colon \tau_1\\ \Gamma \vdash e_2 \colon \tau_2}{\Gamma \vdash \pair{e_1}{e_2} \colon \prodtype{\tau_1}{\tau_2}}\\
  \infer*[left=projLeft]{\Gamma \vdash e : \prodtype{\tau_1}{\tau_2}}{\Gamma \vdash \projl{e} \colon \tau_1} \and
  \infer*[left=projRight]{\Gamma \vdash e : \prodtype{\tau_1}{\tau_2}}{\Gamma \vdash \projr{e} \colon \tau_2}
\end{mathpar}

\begin{thm}[Product Type Inversion]
  If $\Gamma \vdash v : \prodtype{\tau_1}{\tau_2}$ where $v$ is a value, then $v = (v_1, v_2)$ where $v_1$ and $v_2$ are both values and $\Gamma \vdash v_1 : \tau_1$ and $\Gamma \vdash v_2 : \tau_2$.
\end{thm}
\begin{proof}
  By inspecting the possible proofs of $\Gamma \vdash v : \prodtype{\tau_1}{\tau_2}$.
  The only possible rule that can produce a value of this type is the \textsc{pair} rule, which requires that $v = (v_1, v_2)$.
  In order for a pair to be a value, it must be the case that $v_1$ and $v_2$ are values.
  Finally, the \textsc{pair} typing rule gives the required type derivations.
\end{proof}

\section{Sum Types}
\label{sec:sum-types}

\begin{itemize}
\item So far, our only types do not give any choice in terms of what data is provided.
\item In OCaml and Agda, we have ADTs to give a selection of possible types to return from a function.
\item In Agda, we used $\uplus$ as the name for a specific two-choice ADT which we referred to as a \emph{sum type}.
\item We will add this type directly to STLC, rather than the more-general ADTs.
\item However, this does not lose us any power: every (non-recursive) ADT is just a sum of products, which we can write now!
\item In order to test which type we returned, we use a pattern-matching \textsf{case} statement, just as in Agda.
\end{itemize}

\begin{syntax}
  \category[Types]{\tau} \alternative{\ldots} \alternative{\sumtype{\tau_1}{\tau_2}}
  \category[Expressions]{e} \alternative{\ldots} \alternative{\injl{e}} \alternative{\injr{e}} \alternative{\case{e_1}{x}{e_2}{y}{e_3}}
  \category[Values]{v} \alternative{\ldots} \alternative{\injl{v}} \alternative{\injr{v}}
  \category[Evaluation Contexts]{C} \alternative{\ldots} \alternative{\injl{C}} \alternative{\injr{C}} \alternative{\case{C}{x}{e_2}{y}{e_3}}
\end{syntax}

\begin{mathpar}
  \infer*[left=caseLeft]{ }{\case{\injl{v}}{x}{e_2}{y}{e_3} \to_{\text{CBV}} e_2[x \mapsto v]}\and
  \infer*[left=caseRight]{ }{\case{\injr{v}}{x}{e_2}{y}{e_3} \to_{\text{CBV}} e_3[y \mapsto v]}\and
\end{mathpar}

\begin{mathpar}
  \infer*[left=injLeft]{\Gamma \vdash e \colon \tau_1}{\Gamma \vdash \injl{e} \colon \sumtype{\tau_1}{\tau_2}} \and
  \infer*[right=injRight]{\Gamma \vdash e : \tau_2}{\Gamma \vdash \injr{e} \colon \sumtype{\tau_1}{\tau_2}}\\
  \infer*[left=case]{\Gamma \vdash e_1 : \sumtype{\tau_1}{\tau_2}\\ \Gamma, x \colon \tau_1 \vdash e_2 \colon \tau_3\\ \Gamma, y \colon \tau_2 \vdash e_3 \colon \tau_3}{\Gamma \vdash \case{e_1}{x}{e_2}{y}{e_3} \colon \tau_3}
\end{mathpar}

\begin{thm}[Sum Type Inversion]
  If $\Gamma \vdash v \colon \sumtype{\tau_1}{\tau_2}$, then either $v = \injl{v'}$ where $v'$ is a value and $\Gamma \vdash v' \colon \tau_1$, or $v = \injr{v'}$ where $v'$ is a value such that $\Gamma \vdash v' \colon \tau_2$.
\end{thm}
\begin{proof}
  Very similar to the proofs above.
\end{proof}

\section{The \vtype Type}
\label{sec:void-type}

\begin{itemize}
\item We met the void type (named $\bot$) in Agda.
\item It is the type that is never possible to construct.
\item However, it is possible to \emph{destruct}: if someone hands something to you that doesn't exist, you can do whatever you want.
\item We can add it to STLC, but we usually won't: it's not very useful for programming.
\item Nevertheless, it's useful for other applications, and so it is important to talk about.
\end{itemize}

\begin{syntax}
  \category[Types]{\tau} \alternative{\ldots} \alternative{\vtype}
  \category[Expressions]{e} \alternative{\ldots} \alternative{\vcase{e}}
  \category[Evaluation Contexts]{C} \alternative{\ldots} \alternative{\vcase{C}}
\end{syntax}

\begin{mathpar}
  \infer*[left=vcase]{\Gamma \vdash e \colon \vtype}{\Gamma \vdash \vcase{e} \colon \tau}
\end{mathpar}

\begin{itemize}
\item Note that there are no evaluation rules except for evaluation-context rules.
  This comes from the fact that there is never a possible value of type \vtype.
  We'll understand that in more depth soon.
\item The \textsc{vcase} rule says that if you find something of type $\vtype$, you can turn it into any type.
\item This seems dangerous, but type safety won't let it be dangerous.
\end{itemize}

\begin{thm}[Void Inversion]
  There is no closed value~$v$ such that $\vdash v \colon \vtype$.
\end{thm}
\begin{proof}
  Examine all of the possible proofs of $\vdash v : \vtype$.
  Either they are not values, or they do not give rise to the $\vtype$~type. 
\end{proof}

\section{An Aside on Duality}
\label{sec:an-aside-duality}

\begin{itemize}
\item The notion of \emph{duality} is very important in PL Theory.
\item We don't have the language to give a formal definition yet, but we can start to make some observations.
\item First, the notions of injection and projection seem to be related: a projection ``gets something out of'' a product type, while an injection ``puts something in'' a sum type.
\item Second, the notions of pair and case expression are related: a pair creates a product type by providing something of both types, while a case uses a sum type by being prepared to use something of either type.
\item In other words, every way we have of putting together a product type ``corresponds to'' a way of using a sum type, while every way of using a product type ``corresponds to'' a way of putting together a sum type.
\item Because of this, we say that they are \emph{dual} types.
\item Dual types are very closely related: acting as the producer for one is exactly like acting as the consumer for the other.
\item The \utype~and \vtype~types are similarly dual.
\item We will explore this notion of duality more in the next few lectures, and in much more depth in the categorical-semantics unit.
\end{itemize}


\end{document}

%%% Local Variables:
%%% mode: latex
%%% TeX-master: t
%%% TeX-engine: luatex
%%% End:
