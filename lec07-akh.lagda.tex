\documentclass{lecturenotes}

\usepackage[colorlinks,urlcolor=blue]{hyperref}
\usepackage{doi}
\usepackage{xspace}
\usepackage{agda}
\usepackage{fontspec}
\usepackage{enumerate}
\usepackage{mathpartir}
\setsansfont{Fira Code}
\usepackage{newunicodechar}
\newunicodechar{∣}{\ensuremath{\mid}}
\newunicodechar{′}{\ensuremath{{}^\prime}}
\newunicodechar{ˡ}{\ensuremath{{}^{\textsf{l}}}}
\newunicodechar{ʳ}{\ensuremath{{}^{\textsf{r}}}}
\newunicodechar{≤}{\ensuremath{\mathord{\leq}}}
\newunicodechar{≡}{\ensuremath{\mathord{\equiv}}}
\newunicodechar{≐}{\ensuremath{\mathord{\doteq}}}
\newunicodechar{∘}{\ensuremath{\circ}}
\newunicodechar{≃}{\ensuremath{\simeq}}
\newunicodechar{≲}{\ensuremath{\precsim}}
\newunicodechar{⊎}{\ensuremath{\uplus}}

\newcommand{\agdanats}{\textsf{ℕ}\xspace}

\newcommand{\nil}{\ensuremath{\textsf{[]}}}
\newcommand{\cons}{\ensuremath{\mathbin{\textsf{::}}}}
\newcommand{\app}{\ensuremath{\mathbin{\textsf{++}}}}

\title{Negation}
\coursenumber{CSE 410/510}
\coursename{Programming Language Theory}
\lecturenumber{7}
\semester{Spring 2025}
\professor{Professor Andrew K. Hirsch}

\begin{document}
\maketitle

\begin{code}
module lec07-akh where

import Relation.Binary.PropositionalEquality as Eq
open Eq using (_≡_; refl)
open Eq.≡-Reasoning
open import Data.Nat using (ℕ; zero; suc; _+_)
open import Data.Nat.Properties using (+-comm)
open import Isomorphism using (_≃_; _≲_; extensionality; _⇔_)
open import Data.Empty using (⊥; ⊥-elim)
open import Data.Sum using (_⊎_; inj₁; inj₂)
open import Data.Product using (_×_; proj₁; proj₂) renaming (_,_ to ⟨_,_⟩)
\end{code}

\section{Defining Negation}
\label{sec:defining-negation}

\begin{itemize}
\item In logic, if $A$ is true, then $\lnot A$ is false, and vice-versa.
\item In agda, we have to come up with something that has evidence when $A$ does not have any evidence, but does not have evidence when $A$ has evidence.
  \begin{itemize}
  \item The only type we know without evidence is $\bot$.
  \item If you can produce $\bot$ from evidence of an $A$, then $A$ must not have any evidence.
  \item Conversely, if $A$ has no evidence, you can build a function $A \to \bot$ by pattern matching.    
  \end{itemize}
\item We can formalize this as follows:
\begin{code}
¬_ : Set → Set
¬ A = A → ⊥

infix 3 ¬_    
\end{code}
\item It is impossible to have evidence of both $A$ and $\lnot A$.
\begin{code}
¬-elim : ∀ {A : Set} →
  ¬ A →
    A →
  ------
    ⊥
¬-elim ¬x x = ¬x x  
\end{code}
\item Moreover, if we have evidence of $A$, it's impossible to get evidence of $\lnot A$.
\begin{code}
¬¬-intro : ∀ {A : Set} →
     A →
  -------
   ¬ ¬ A
¬¬-intro x = λ {¬x → ¬x x}    
\end{code}
\item That's exactly what we wanted to say that this is evidence of negation!
\item One logical principle that can be very useful is \emph{contraposition}: if $A$ implies $B$, then $\lnot B$ implies $\lnot A$.
  Note the switch!
  It's easy to prove this in agda.
\end{itemize}

\section{Proofs of Negation}
\label{sec:proofs-negation}
\begin{itemize}
\item If we assume function extensionality, then the fact that there are no proofs of $\bot$ means that proofs of negation are unique.
\item For instance, there is exactly one proof of $\bot \to \bot$.
\item There are two obvious ways to prove such a thing, but they are equal.
\begin{code}
id-⊥ : ⊥ → ⊥
id-⊥ x = x

id-⊥′ : ⊥ → ⊥
id-⊥′ ()

id≡id′ : id-⊥ ≡ id-⊥′
id≡id′ = extensionality (λ())    
\end{code}
\item In fact, for any proposition there is at most one proof of negation.
  \begin{itemize}
  \item To see this, imagine that I have two proofs~$f$ and~$g$ of $\lnot \varphi$.
  \item Then, imagine that I have evidence~$x$ of $\varphi$.
  \item We need to prove that $f(x)$ and $g(x)$ are equal.
  \item We can prove this by ex~falso, since $f(x) : \bot$!
  \end{itemize}
\item We can formalize this as follows:
\begin{code}
assimilation : ∀ {A : Set} (¬x ¬y : ¬ A) → ¬x ≡ ¬y
assimilation ¬x ¬y = extensionality (λ x → ⊥-elim (¬x x))    
\end{code}
\end{itemize}

\section{Inequality}
\label{sec:inequality}

\begin{itemize}
\item We can easily define inequality using negation:
\begin{code}
_≢_ : ∀ {A : Set} → A → A → Set
t ≢ u = ¬ (t ≡ u)    
\end{code}
\item Proving concrete things inequal can be done via absurd patterns.
\begin{code}
_ : 1 ≢ 2
_ = λ ()    
\end{code}
\begin{itemize}
\item This is the first time we've seen an absurd pattern in a lambda.
  It is not any different any other place we've seen it.
\end{itemize}
\item In fact, agda can see that anything with different constructors is different, without fuss.
\begin{code}
peano : ∀ {m : ℕ} → zero ≢ suc m
peano = λ()    
\end{code}
\item However, not everything is like this.
  If we try to prove $\forall m.\,m \not\equiv (m + 1)$ with \textsf{λ()}, agda will not accept it.
  After all, it doesn't know whether they have different constructors!
\item However, we can use +-comm to convince agda that they do.
\begin{code}
plus-inequal : ∀ {m : ℕ} → m ≢ (m + 1)
plus-inequal {m} rewrite +-comm m 1 = λ()    
\end{code}
\end{itemize}

\pagebreak
\section{Double Negation}
\label{sec:double-negation}

\begin{itemize}
\item One important operator in logic is the \emph{double negation} operator.
  It's just two applications of negation.
\begin{code}
¬¬ : Set → Set
¬¬ A = ¬ ¬ A    
\end{code}
\item In English, negating something twice makes it positive again.
  \begin{itemize}
  \item ``It's not not good'' means it's bad.
  \end{itemize}
\item This is \emph{double-negation elimination}.
\item However, this doesn't work in agda. (Why not?)
\item If it was and you knew that $\lnot A$ had no evidence, then you'd have to be able to find evidence of $A$.
\item But that's just not how it works!
  \begin{itemize}
  \item This is a really important point in science: just because you've found there is no evidence something isn't there, doesn't mean you have evidence it is there.
  \end{itemize}
\item The idea that you cannot magic up evidence is the core idea of \emph{intuitionistic} or \emph{constructive} logic, as opposed to \emph{classical} logic.
\item This also means that intuitionistic logic rejects \emph{the law of the excluded middle} (which states that $A \lor \lnot A$ holds for any $A$).
  \begin{itemize}
  \item Otherwise, we're magicking up evidence for either $A$ or $\lnot A$.
  \item This is related to the \emph{decision problem}: there are true-or-false questions that no program can reliably answer.
  \end{itemize}
\item One consequence of this: if you have evidence of $A \lor B$, then you can read from that evidence \emph{which} of $A$ or $B$ holds.
\item While we can't get rid of double negation in general, we can get rid of it on top of another negation.
\begin{code}
¬¬¬-elim : ∀ {A : Set} →
  ¬¬ (¬ A) →
  ---------
    ¬ A
¬¬¬-elim ¬¬¬x x = ¬¬¬x (¬¬-intro x)    
\end{code}
\item Moreover, we can do our reasoning under double negation:
\begin{code}
¬¬-bind : ∀ {A B : Set} →
   (A → ¬¬ B) →
     ¬¬ A →
  ------------
     ¬¬ B
       
¬¬-bind f ¬¬x ¬y = ¬¬x (λ {x → f x ¬y})    
\end{code}
\begin{itemize}
\item Those of you that took CSE~505 with me will recognize that this makes $\mathord{\lnot}\mathord{\lnot}$ a \emph{monad}.
\end{itemize}
\newpage
\item In particular, we can do \emph{classical} reasoning under $\mathord{\lnot}\mathord{\lnot}$.
\begin{code}
em-irrefutable : ∀ {A : Set} → ¬¬ (A ⊎ ¬ A)
em-irrefutable = λ {k → k (inj₂ (λ {x → k (inj₁ x)}))}
\end{code}
  \begin{itemize}
  \item This is hard to understand without walking through it.
    Luckily, we have agda to help!
  \item We start out by creating a hole:

    \begin{center}\textsf{em-irrefutable k = ?}\end{center}
  \item We now have $k \colon \textsf{¬ (A ⊎ ¬ A)}$, and we need to prove $\bot$.
  \item The only chance we have is applying \textsf{k}.

    \begin{center}\textsf{em-irrefutable k = k ?}\end{center}
  \item Now, we need to prove \textsf{A ⊎ ¬ A}.
    We don't have any chance of proving $A$, so let's take the second option.

    \begin{center}\textsf{em-irrefutable k = k (inj₂ (λ {x → ?}))}\end{center}
  \item Now we have $x \colon \textsf{A}$, and we need to prove $\bot$ again.
    Yet again, the only chance we have is to apply $k$.

    \begin{center}\textsf{em-irrefutable = λ {k → k (inj₂ (λ {x → k ?}))}}\end{center}
  \item Now we need to prove \textsf{A ⊎ ¬ A} again.
    This may be worrying: it feels like we're caught in a loop!
    But now, we know that \textsf{A} holds, so we can take the other option.

    \begin{center}\textsf{em-irrefutable = λ {k → k (inj₂ (λ {x → k (inj₁ x)}))}}\end{center}
  \item And we're done!
  \end{itemize}
\end{itemize}

\end{document}

%%% Local Variables:
%%% mode: latex
%%% TeX-master: t
%%% TeX-engine: luatex
%%% TeX-command-default: "Make"
%%% End:
