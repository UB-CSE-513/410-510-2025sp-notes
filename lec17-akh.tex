\documentclass{lecturenotes}

\usepackage[colorlinks,urlcolor=blue]{hyperref}
\usepackage{doi}
\usepackage{xspace}
\usepackage{fontspec}
\usepackage{enumerate}
\usepackage{mathpartir}
\usepackage{pl-syntax/pl-syntax}
\usepackage{forest}
\usepackage{stmaryrd}
\usepackage{epigraph}
\usepackage{xspace}

\setsansfont{Fira Code}

\newcommand{\abs}[2]{\ensuremath{\lambda #1.\,#2}}
\newcommand{\tabs}[3]{\ensuremath{\lambda #1 \colon #2.\,#3}}
\newcommand{\dbabs}[1]{\ensuremath{\lambda.\,#1}}
\newcommand{\dbind}[1]{\ensuremath{\text{\textasciigrave}#1}}
\newcommand{\app}[2]{\ensuremath{#1\;#2}}
\newcommand{\utype}{\textsf{unit}\xspace}
\newcommand{\unit}{\ensuremath{\textsf{(}\mkern0.5mu\textsf{)}}}
\newcommand{\prodtype}[2]{\ensuremath{#1 \times #2}}
\newcommand{\pair}[2]{\ensuremath{(#1, #2)}}
\newcommand{\projl}[1]{\ensuremath{\pi_1\mkern2mu#1}}
\newcommand{\projr}[1]{\ensuremath{\pi_2\mkern3mu#1}}
\newcommand{\sumtype}[2]{\ensuremath{#1 + #2}}
\newcommand{\injl}[1]{\ensuremath{\textsf{inj}_1\mkern2mu#1}}
\newcommand{\injr}[1]{\ensuremath{\textsf{inj}_2\mkern3mu#1}}
\newcommand{\case}[5]{\ensuremath{\textsf{case}\mkern5mu#1\mkern5mu\textsf{of}\mkern5mu\injl{#2} \Rightarrow #3;\mkern5mu\injr{#4} \Rightarrow #5\mkern5mu\textsf{end}}}
\newcommand{\vtype}{\textsf{void}\xspace}
\newcommand{\vcase}[1]{\ensuremath{\textsf{case}\mkern5mu#1\mkern5mu\textsf{of}\mkern5mu\textsf{end}}}

\newcommand{\FV}{\text{FV}}
\newcommand{\BV}{\text{BV}}

\title{Type System Organization}
\coursenumber{CSE 410/510}
\coursename{Programming Language Theory}
\lecturenumber{17}
\semester{Spring 2025}
\professor{Professor Andrew K. Hirsch}

\begin{document}
\maketitle

\begin{itemize}
\item Last week, we looked at \utype, \vtype, product types, and sum types.
\item These are familiar from languages like OCaml and Agda, though their form may be less familiar.
\item We focused on what it is that these types represent and how they worked from a programmer's perspective.
  Today, we draw out some principles that we can use to organize type systems and programming languages.
  These principles will be applied over and over again for the rest of the semester.
\end{itemize}

\section{Introduction and Elimination Forms}
\label{sec:intr-elim-forms}

Recall the type system for STLC + \utype, \vtype, products, and sums:
$$\begin{array}{c|c}
  \multicolumn{2}{c}{\infer*[left=Axiom]{x \colon \tau \in \Gamma}{\Gamma \vdash x \colon \tau}}\\[2em]
  \infer*[left=arrI]{\Gamma, x \colon \tau_1 \vdash e \colon \tau_2}{\Gamma \vdash \tabs{x}{\tau_1}{e} \colon \tau_1 \to \tau_2} & \infer*[right=arrE]{\Gamma \vdash f \colon \tau_1 \to \tau_2\\ \Gamma \vdash e \colon \tau_1}{\Gamma \vdash \app{f}{e} \colon \tau_2}\\[2em]
  \infer*[left=unitI]{ }{\Gamma \vdash \unit \colon \utype} & \\[2em]
                                                                             & \infer*[right=voidE]{\Gamma \vdash e \colon \vtype}{\Gamma \vdash \vcase{e} \colon \tau}\\[2em]
  \infer*[left=prodI]{\Gamma \vdash e_1 \colon \tau_1\\ \Gamma \vdash e_2 \colon \tau_2}{\Gamma \vdash (e_1, e_2) \colon \prodtype{\tau_1}{\tau_2}} & \infer*[right=prodE${}_1$]{\Gamma \vdash e \colon \prodtype{\tau_1}{\tau_2}}{\Gamma \vdash \projl{e} \colon \tau_1} \hspace{1em} \infer*[right=prodE${}_2$]{\Gamma \vdash e \colon \prodtype{\tau_1}{\tau_2}}{\Gamma \vdash \projr{e} \colon \tau_2}\\[2em]
  \infer*[left=sumI${}_1$]{\Gamma \vdash e \colon \tau_1}{\Gamma \vdash \injl{e} \colon \sumtype{\tau_1}{\tau_2}} \hspace{1em} \infer*[left=sumI${}_2$]{\Gamma \vdash e \colon \tau_2}{\Gamma \vdash \injr{e} \colon \sumtype{\tau_1}{\tau_2}} &\infer*[right=sumE]{\Gamma \vdash e_1 \colon \sumtype{\tau_1}{\tau_2}\\ \Gamma, x \colon \tau_1 \vdash e_2 \colon \tau_3\\ \Gamma, y \colon \tau_2 \vdash e_3 \colon \tau_3}{\Gamma \vdash \case{e_1}{x}{e_2}{y}{e_3} \colon \tau_3}\\
\end{array}$$

\vspace{2em}
\begin{itemize}
\item As you can see, I have imposed some structure on the rules.
\item The axiom rule is special: it doesn't tell us anything about a particular type.
  Instead, it tells us how to use scope.
\item Every other type constructor has some number of \emph{introduction} rules and some number of \emph{elimination} rules.
  \begin{itemize}
  \item Introduction rules tell us how to \emph{create} something of a type.
  \item Elimination rules tell us how to \emph{use} (or \emph{consume}) something of a type.
  \end{itemize}
\item The arrow type constructor has one of each:
  \begin{itemize}
  \item The introduction rule~\textsc{arrI} tells us how to create an arrow: add a new variable to scope and then show that you can create a return value.
  \item The elimination rule~\textsc{arrE} tells us how to use an arrow: call the function with an argument to bind the variable to.
  \end{itemize}
\item The \utype~type has one introduction rule and no elimination rules:
  \begin{itemize}
  \item The introduction rule~\textsc{unitI} tells us how to create an~\utype: simply provide the value $\unit$.
  \item The lack of an elimination rule says that there is no way to get information out of a \utype.
  \end{itemize}
\item The \vtype~type has no introduction rules but has one elimination rule:
  \begin{itemize}
  \item The lack of an introduction rule says that there is no way to create a \vtype.
  \item The elimination rule~\textsc{voidE} tells us that you can get arbitrary information out of any \vtype you can find.
  \end{itemize}
\item The product type constructor has one introduction rule and two elimination rules:
  \begin{itemize}
  \item The introduction rule~\textsc{prodI} says that to create a tuple, one must provide something of each of the two types.
  \item The elimination rules~\textsc{prodE${}_1$} and~\textsc{prodE${}_2$} tell us how to get each of those ``somethings'' out of the tuple.
  \end{itemize}
\item The sum type constructor has two introduction rules and one elimination rule:
  \begin{itemize}
  \item The introduction rules~\textsc{sumI${}_1$} and~\textsc{sumI${}_2$} tell us that to create something of sum type, you must provide something of either of the two types.
  \item The elimination rule~\textsc{sumE} tells us that we can get information out of a sum type by being willing to consume either type.
  \end{itemize}
\item From now on, when we add a new type constructor to a language, we will talk about its introduction and elimination rules.
  We will tend to name introduction rules with an \textsc{I} and elimination rules with a \textsc{E}.
\end{itemize}

\section{$\beta$ Laws}
\label{sec:beta-laws}

\begin{itemize}
\item This discipline does more than impose a structure on a type system and provide a guide for how to add a new type.
\item Instead, important equational laws come from this organization.
\item We start by looking at $\beta$ laws.
\item Consider the $\beta$ law for functions that we have seen since last week:
  $$\app{(\tabs{x}{\tau_1}{e_1})}{e_2} \equiv_\beta e_1[x \mapsto e_2]$$
\item Now let's see what the typing proof for the left-hand side must be:
  $$
  \infer*[left=arrE]{\infer*[left=arrI]{\Gamma, x \colon \tau_1 \vdash e_1 \colon \tau_2}{\Gamma \vdash \tabs{x}{\tau_1}{e_1} \colon \tau_1 \to \tau_2}\\ \Gamma \vdash e_2 \colon \tau_1}{\Gamma \vdash \app{(\tabs{x}{\tau_1}{e_1})}{e_2} \vdash \tau_2}
  $$
\item Notice something about the way this typing derivation works: there's an introduction rule on top and an elimination rule immediately underneath it.
\item If we follow the same pattern for product types, we get the following:
  $$
  \infer*[left=projE${}_1$]{\infer*[left=prodI]{\Gamma \vdash e_1 : \tau_1\\ \Gamma \vdash e_2 \colon \tau_2}{\Gamma \vdash \pair{e_1}{e_2} \colon \prodtype{\tau_1}{\tau_2}}}{\Gamma \vdash \projl{\pair{e_1}{e_2}} \colon \tau_1}
  $$
\item \textbf{Stop and Think:} What should the corresponding equation be?
\item There's only one possible answer that has the right type:
  $$\projl{\pair{e_1}{e_2}} \equiv_\beta e_1$$
\item There are actually two $\beta$~rules for products.
  The other is as follows: $$\projr{\pair{e_1}{e_2}} \equiv_\beta e_2$$
\item Sum types similarly have two rules.
  We can see the first one clearly by considering the following typing derivation:
  $$
  \infer*[left=sumE]{\infer*[left=sumI${}_1$]{\Gamma \vdash e_1 \colon \tau_1}{\Gamma \vdash \injl{e_1} \colon \sumtype{\tau_1}{\tau_2}}\\ \Gamma, x \colon \tau_1 \vdash e_2 \colon \tau_3\\ \Gamma, y \colon \tau_2 \vdash e_3 \colon \tau_3}{\case{\injl{e_1}}{x}{e_2}{y}{e_3}}
  $$
\item The corresponding $\beta$ rule is:
  $$\case{\injl{e_1}}{x}{e_2}{y}{e_3} \equiv_\beta e_2[x \mapsto e_1]$$
\item Alternatively, we can use the other injection:
  $$\case{\injr{e_1}}{x}{e_2}{y}{e_3} \equiv_\beta e_3[y \mapsto e_1]$$
\item Since \utype has no elimination rule, it has no associated $\beta$ laws.
\item We could apply the same reasoning to \vtype.
  However, it's useful to give it the following $\beta$ law:
  $$C[\vcase{e}] \equiv_\beta \vcase{e}$$
\item Unlike other $\beta$ rules, this doesn't get rid of the elimination rule, because there's no introduction rule for it to ``react'' with.
  Instead, it just says that the elimination rule ``absorbs'' everything around it.
\item You could argue that it's not a ``real'' $\beta$ rule, but it traditionally is considered one.
\end{itemize}

\subsection{$\beta$ Reductions}
\label{sec:beta-reductions}

\begin{itemize}
\item So far, we've talked all about $\beta$ \emph{equivalences}.
\item However, we often care instead about $\beta$ \emph{reductions}.
  \begin{itemize}
  \item These reductions themselves are then limited to something like CBV or CBN to make small-step operational semantics.
  \end{itemize}
\item $\beta$ equivalences equate a term with an introduction rule directly above an elimination rule to a term with no introduction or elimination rule.
\item $\beta$ reductions move from the term with the rules to the term without.
\item So far, everything we've given is for full $\beta$, but can be restricted to CBV by forcing everything in the introduction form to be a value.
\end{itemize}

\section{$\eta$ Laws}
\label{sec:eta-laws}

\begin{itemize}
\item Just like $\beta$ equivalences, $\eta$ equivalences fall out from the organization of types.
\item Let's again look at the $\eta$ equivalence we had before for pure $\lambda$~calculus:
  $$e \equiv_\eta \abs{x}{(app{e}{x})}$$
\item If we type the right-hand side, we get the following:
  $$
  \infer*[left=arrI]{
    \infer*[left=arrE]{
      \Gamma, x \colon \tau_1 \vdash e \colon \tau_1 \to \tau_2\\
      \infer*[left=Axiom]{ }{\Gamma, x \colon \tau_1 \vdash x \colon \tau_1}
    }{
      \Gamma, x \colon \tau_1 \vdash \app{e}{x} \colon \tau_2
    }
  }{
    \Gamma \vdash \tabs{x}{\tau_1}{(\app{e}{x})} \colon \tau_1 \to \tau_2
  }
  $$
\item Whereas before we had an elimination rule directly under an introduction rule, now we have an introduction rule directly under an elimination rule.
\item Again, this is the common style for $\eta$ laws.
\item For products, we get the following (single) $\eta$ law:
  $$e \equiv_\eta \pair{\projl{e}}{\projr{e}}$$
\item If we give a type to the right-hand side, we get:
  $$
  \infer*[left=prodI]{
    \infer*[left=prodE${}_1$]{\Gamma \vdash e \colon \tau_1 \times \tau_2}{\Gamma \vdash \projl{e} \colon \tau_1}\\
    \infer*[left=prodE${}_2$]{\Gamma \vdash e \colon \tau_1 \times \tau_2}{\Gamma \vdash \projr{e} \colon \tau_2}
  }{
    \Gamma \vdash \pair{\projl{e}}{\projr{e}} \colon \prodtype{\tau_1}{\tau_2}
  }
  $$
\item For sums the story is a bit stranger.
  We don't use a sum by simply returning the data inside of it; instead, we explicitly substitute the data inside of it into some \emph{continuation}.
  However, otherwise, it's exactly the same.
$$e \equiv_\eta \case{e}{x}{\injl{x}}{y}{\injr{y}}$$
\item Just like there is no $\beta$ rule for \utype, there is no $\eta$ rule for \vtype.
\item However, just like there is a pseudo-$\beta$ rule for \vtype, there is a similar pseudo-$\eta$ rule for \utype:
  $$e \equiv_\eta \unit$$
\item It is important to know that this holds for any $e$ \emph{which has unit type}.
\item In fact, this rule throws into relief something important: \textbf{$\eta$ and $\beta$ rules are typed}.
\item We could ignore this in untyped $\lambda$ calculus and even simply-typed lambda calculus, because everything was a function.
\item Now, however, it's extremely important to keep types in mind.
\end{itemize}

\end{document}

%%% Local Variables:
%%% mode: latex
%%% TeX-master: t
%%% TeX-engine: luatex
%%% End:
