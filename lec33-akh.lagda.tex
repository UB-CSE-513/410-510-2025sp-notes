\documentclass{lecturenotes}

\usepackage[colorlinks,urlcolor=blue]{hyperref}
\usepackage{doi}
\usepackage{xspace}
\usepackage{fontspec}
\usepackage{enumerate}
\usepackage{mathpartir}
\usepackage{pl-syntax/pl-syntax}
\usepackage{forest}
\usepackage{stmaryrd}
\usepackage{epigraph}
\usepackage{xspace}
\usepackage{bbm}
\usepackage{tikz-cd}
\usepackage{unicode-math}

\setsansfont{Fira Code}
\setmathfont{AsanaMath}

\newcommand{\abs}[2]{\ensuremath{\lambda #1.\,#2}}
\newcommand{\tabs}[3]{\ensuremath{\lambda #1 \colon #2.\,#3}}
\newcommand{\dbabs}[1]{\ensuremath{\lambda.\,#1}}
\newcommand{\dbind}[1]{\ensuremath{\text{\textasciigrave}#1}}
\newcommand{\app}[2]{\ensuremath{#1\;#2}}
\newcommand{\utype}{\textsf{unit}\xspace}
\newcommand{\unit}{\ensuremath{\textsf{(}\mkern0.5mu\textsf{)}}}
\newcommand{\prodtype}[2]{\ensuremath{#1 \times #2}}
\newcommand{\pair}[2]{\ensuremath{(#1, #2)}}
\newcommand{\projl}[1]{\ensuremath{\pi_1\mkern2mu#1}}
\newcommand{\projr}[1]{\ensuremath{\pi_2\mkern3mu#1}}
\newcommand{\sumtype}[2]{\ensuremath{#1 + #2}}
\newcommand{\injl}[1]{\ensuremath{\textsf{inj}_1\mkern2mu#1}}
\newcommand{\injr}[1]{\ensuremath{\textsf{inj}_2\mkern3mu#1}}
\newcommand{\case}[5]{\ensuremath{\textsf{case}\mkern5mu#1\mkern5mu\textsf{of}\mkern5mu\injl{#2} \Rightarrow #3;\mkern5mu\injr{#4} \Rightarrow #5\mkern5mu\textsf{end}}}
\newcommand{\vtype}{\textsf{void}\xspace}
\newcommand{\vcase}[1]{\ensuremath{\textsf{case}\mkern5mu#1\mkern5mu\textsf{of}\mkern5mu\textsf{end}}}
\newcommand{\rectype}[2]{\ensuremath{\mu #1.\,#2}}
\newcommand{\roll}[1]{\textsf{roll}\mkern2mu#1}
\newcommand{\unroll}[1]{\textsf{unroll}\mkern2mu#1}
\newcommand{\fatype}[2]{\ensuremath{\forall #1.\,#2}}
\newcommand{\Abs}[2]{\Lambda #1.\,#2}
\newcommand{\App}[2]{#1\;[#2]}
\newcommand{\extype}[2]{\ensuremath{\exists #1.\,#2}}
\newcommand{\pack}[3]{\ensuremath{\textsf{pack}\mkern5mu#1\mathrel{\textsf{as}}#2\mathrel{\textsf{in}}#3}}
\newcommand{\unpack}[4]{\ensuremath{\textsf{unpack}\mkern5mu#1\mathrel{\textsf{as}} #2, #3 \mathrel{\textsf{in}} #4}}
\newcommand{\ltype}[1]{\ensuremath{\textsf{list}\mkern3mu#1}}
\newcommand{\nillist}{\ensuremath{[\,]}}
\newcommand{\conslist}[2]{\ensuremath{#1 \mathop{::} #2}}
\newcommand{\lcase}[5]{\ensuremath{\textsf{case}\mkern5mu#1\mkern5mu\textsf{of}\mkern5mu\nillist\Rightarrow#2; \mkern5mu\conslist{#3}{#4} \Rightarrow #5\mkern5mu\textsf{end}}}
\newcommand{\logn}[1]{\ensuremath{\textsf{log}\mkern3mu#1}}

\newcommand{\capture}[1]{\ensuremath{\lfloor #1 \rfloor}}
\newcommand{\bind}[1]{\ensuremath{\textsf{bind}(#1)}}

\newcommand{\pureeff}{\textsf{pure}\xspace}
\newcommand{\logeff}{\textsf{log}\xspace}

\newcommand{\FV}{\text{FV}}
\newcommand{\BV}{\text{BV}}

\newcommand{\toform}[1]{\ensuremath{\lceil #1 \rceil}}
\newcommand{\totype}[1]{\ensuremath{\lfloor #1 \rfloor}}

\newcommand{\neutral}[1]{#1\;\text{ne}}
\newcommand{\nf}[1]{#1\;\text{nf}}

\newcommand{\subtype}{\ensuremath{\mathrel{\mathord{<}\mathord{:}}}}

\newcommand{\pub}{\text{public}}
\newcommand{\priv}{\text{secret}}

\newcommand{\at}{\ensuremath{\mathrel{@}}}

\newcommand{\obj}[1]{\ensuremath{\mathcal{O}(#1)}}
\renewcommand{\hom}[3][]{\ensuremath{\text{Hom}_{#1}(#2, #3)}}
\newcommand{\id}[1][]{\ensuremath{\mathbbm{1}_{#1}}}

\newcommand{\Set}{\textbf{Set}\xspace}
\newcommand{\Rel}{\textbf{Rel}\xspace}
\newcommand{\Type}{\textbf{Type}\xspace}
\newcommand{\Cat}{\textbf{Cat}\xspace}

\newcommand{\op}[1]{\ensuremath{{#1}^{\text{op}}}}

\newcommand{\prodmor}[2]{\ensuremath{\langle #1, #2 \rangle}}
\newcommand{\inl}{\text{inl}\xspace}
\newcommand{\inr}{\text{inr}\xspace}
\newcommand{\coprodmor}[2]{\ensuremath{[ #1, #2 ]}}

\title{Monads and Effects}
\coursenumber{CSE 410/510}
\coursename{Programming Language Theory}
\lecturenumber{33}
\semester{Spring 2025}
\professor{Professor Andrew K. Hirsch}

\begin{document}
\maketitle

\begin{itemize}
\item Now that we have the basics of category theory under our belt, let's begin to look at something category theory can teach us about programming languages.
\item We will consider how to organize the semantics of \emph{type-and-effect} systems, that track both the type of a program and what effects it can perform.
\item Up until now, we have mostly focused on \emph{pure} programs: they only receive inputs and compute outputs without having any side effects like printing to the screen or launching the missiles.
\item We can easily add an effect; for instance, let's focus on a $\lambda$~calculus that can append information to a log of numbers.
\end{itemize}

\section*{An Effectful Calculus}

\begin{syntax}
  \category[Effects]{\varepsilon} \alternative{\pureeff} \alternative{\logeff}
  \category[Types]{\tau} \alternative{\utype} \alternative{\mathbb{N}}  \alternative{\ltype{\tau}} \alternative{\prodtype{\tau_1}{\tau_2}} \alternative{\sumtype{\tau_1}{\tau_2}} \alternative{\tau_1 \xrightarrow{\varepsilon} \tau_2}
  \category[Expressions]{e} \alternative{x} \alternative{\unit} \alternative{\tilde{n}} \alternative{\logn{e}}\\
  \alternative{\nillist} \alternative{\conslist{e_1}{e_2}} \alternative{\lcase{e}{e_1}{x}{y}{e_2}}\\
  \alternative{\pair{e_1}{e_2}} \alternative{\projl{e}} \alternative{\projr{e}}\\
  \alternative{\injl{e}} \alternative{\injr{e}} \alternative{\case{e}{x}{e_1}{y}{e_2}}\\
  \alternative{\tabs{x}{\tau}{e}} \alternative{\app{e_1}{e_2}}
  \category[Values]{v} \alternative{\unit} \alternative{\tilde{n}} \alternative{\nillist} \alternative{\conslist{v_1}{v_2}} \alternative{\pair{v_1}{v_2}} \alternative{\injl{v}} \alternative{\injr{v}} \alternative{\tabs{x}{\tau}{e}}
  \category[Evaluation Contexts]{\mathcal{C}} \alternative{[\cdot]} \alternative{\logn{\mathcal{C}}}\\
  \alternative{\conslist{\mathcal{C}}{e}} \alternative{\conslist{v}{\mathcal{C}}} \alternative{\lcase{\mathcal{C}}{e_1}{x}{y}{e_2}}\\
  \alternative{\pair{\mathcal{C}}{e}} \alternative{\pair{v}{\mathcal{C}}} \alternative{\projl{\mathcal{C}}} \alternative{\projr{\mathcal{C}}}\\
  \alternative{\injl{\mathcal{C}}} \alternative{\injr{\mathcal{C}}} \alternative{\case{\mathcal{C}}{x}{e_1}{y}{e_2}}\\
  \alternative{\app{\mathcal{C}}{e}} \alternative{\app{v}{\mathcal{C}}}
\end{syntax}

\begin{itemize}
\item Note that we're keeping track of \emph{effects} as well at types.
\item The effect \logeff means that this program \textbf{might} log some data, while \pureeff means that it \textbf{will definitely not} log any data.
\item Function values are pure, even when their bodies are not.
  Intuitively, a function does not log anything, but when its is applied it may log anything.
  As such, we need to track the potential effect of a function in the type; we write $\tau_1 \xrightarrow{\varepsilon} \tau_2$ for the type of a function which takes an argument of type~$\tau_1$ and produces a result of type~$\tau_2$, potentially preforming the effect~$\varepsilon$.
\item In the operational semantics, we need to keep track not only of data, but also of the current log.
  Since we log numbers, we keep track of a list of numbers as we go, appending to that list when we log something.
\end{itemize}

\begin{mathpar}
  \infer{e_1, \ell \to e_2, \ell'}{\mathcal{C}[e_1], \ell \to \mathcal{C}[e_2], \ell'} \and
  \infer{}{\logn{\tilde{n}}, \ell \to \unit, n\ell}\\
  \infer{}{\lcase{\nillist}{e_1}{x}{y}{e_2}, \ell \to e_1, \ell}\and
  \infer{}{\lcase{(\conslist{v_1}{v_2})}{e_1}{x}{y}{e_2}, \ell \to e_2[x \mapsto v_1, y \mapsto v_2], \ell} \\
  \infer{}{\projl{\pair{v_1}{v_2}}, \ell \to v_1, \ell} \and
  \infer{}{\projr{\pair{v_1}{v_2}}, \ell \to v_2, \ell} \\
  \infer{}{\case{(\injl{v})}{x}{e_1}{y}{e_2}, \ell \to e_1[x \mapsto v], \ell} \and
  \infer{}{\case{(\injr{v})}{x}{e_1}{y}{e_2}, \ell \to e_2[y \mapsto v], \ell} \\
  \infer{}{\app{(\tabs{x}{\tau}{e})}{v}, \ell \to e[x \mapsto v], \ell}
\end{mathpar}

\begin{itemize}
\item We write $\Gamma \vdash e : \tau \mid \varepsilon$ to mean ``in context~$\Gamma$, the program~$e$ will eventually compute a~$\tau$, potentially performing effect~$\varepsilon$.''
\item This is called a \emph{type-and-effect system}, since it tracks both the types of programs and the effects that they may have.
\item Note that nothing \emph{forces} a program to be pure: a pure program can be considered to be a program that potentially logs data.
\item We can also combine two effects: $\varepsilon_1 \cdot \varepsilon_2$ is the effect that results from running both effect~$\varepsilon_1$ and the effect~$\varepsilon_2$.
  We can define this as follows:
  $$
  \varepsilon_1 \cdot \varepsilon_2 \triangleq \left\{\begin{array}{ll} \pureeff & \text{if}~\varepsilon_1 = \varepsilon_2 = \pureeff\\ \logeff & \text{otherwise}\end{array}\right.
  $$
\end{itemize}

\begin{mathparpagebreakable}
  \infer*[left=Axiom]{x : \tau \in \Gamma}{\Gamma \vdash x : \tau \mid \varepsilon}\\
  \infer*[left=UnitI]{ }{\Gamma \vdash \unit : \utype \mid \varepsilon}\\
  \infer*[left=Lit]{ }{\Gamma \vdash \tilde{n} : \mathbb{N} \mid \varepsilon}\and
  \infer*[right=Log]{\Gamma \vdash e : \mathbb{N} \mid\varepsilon}{\Gamma \vdash \logn{e} : \utype \mid \logeff}\\
  \infer*[left=ListI$_1$]{ }{\Gamma \vdash \nillist : \ltype{\tau} \mid \varepsilon}\and
  \infer*[left=ListI$_2$]{\Gamma \vdash e_1 : \tau \mid \varepsilon_1\\ \Gamma \vdash e_2 : \ltype{\tau} \mid \varepsilon_2}{\Gamma \vdash \conslist{e_1}{e_2} : \ltype{\tau} \mid \varepsilon_1 \cdot \varepsilon_2}\and
  \infer*[right=ListE]{\Gamma \vdash e : \ltype{\tau} \mid \varepsilon_1\\ \Gamma \vdash e_1 : \sigma \mid \varepsilon_2\\ \Gamma, x : \tau, y : \ltype{\tau} \vdash e_2 : \sigma \mid \varepsilon_3}{\Gamma \vdash \lcase{e}{e_1}{x}{y}{e_2} \mid \varepsilon_1 \cdot \varepsilon_2 \cdot \varepsilon_3}\\
  \infer*[left=$\times$-I]{\Gamma \vdash e_1 : \tau_1 \mid \varepsilon_1 \\ \Gamma \vdash e_2 : \tau_2 \mid \varepsilon_2}{\Gamma \vdash \pair{e_1}{e_2} : \prodtype{\tau_1}{\tau_2} \mid \varepsilon_1 \cdot \varepsilon_2}\and
  \infer*[right=$\times$-E$_1$]{\Gamma \vdash e : \prodtype{\tau_1}{\tau_2} \mid \varepsilon}{\Gamma \vdash \projl{e} : \tau_1 \mid \varepsilon} \and
  \infer*[right=$\times$-E$_2$]{\Gamma \vdash e : \prodtype{\tau_1}{\tau_2} \mid \varepsilon}{\Gamma \vdash \projr{e} : \tau_2 \mid \varepsilon} \\
  \infer*[left=$+$-I$_1$]{\Gamma \vdash e : \tau_1 \mid \varepsilon}{\Gamma \vdash \injl{e} : \sumtype{\tau_1}{\tau_2} \mid \varepsilon} \and
  \infer*[left=$+$-I$_2$]{\Gamma \vdash e : \tau_2 \mid \varepsilon}{\Gamma \vdash \injr{e} : \sumtype{\tau_1}{\tau_2} \mid \varepsilon} \and
  \infer*[right=$+$-E]{\Gamma \vdash e : \sumtype{\tau_1}{\tau_2} \mid \varepsilon_1\\ \Gamma, x : \tau_1 \vdash e_1 : \sigma \mid \varepsilon_2\\ \Gamma, y : \tau_2 \vdash e_2 : \sigma \mid \varepsilon_3}{\Gamma \vdash \case{e}{x}{e_1}{y}{e_2} : \sigma \mid \varepsilon_1 \cdot \varepsilon_2 \cdot \varepsilon_3}\\
  \infer*[left=$\to$-I]{\Gamma, x : \tau_1 \vdash e : \tau_2 \mid \varepsilon}{\Gamma \vdash \tabs{x}{\tau_1}{e} : \tau_1 \xrightarrow{\varepsilon} \tau_2 \mid \varepsilon}\and
  \infer*[right=$\to$-E]{\Gamma \vdash f : \tau_1 \xrightarrow{\varepsilon_f} \tau_2 \mid \varepsilon_1\\ \Gamma \vdash a : \tau_1 \mid \varepsilon_2}{\Gamma \vdash \app{f}{a} : \tau_2 \mid \varepsilon_1 \cdot \varepsilon_2 \cdot \varepsilon_f}
\end{mathparpagebreakable}

\begin{thm}[Progress]
  If $\vdash e : \tau \mid \varepsilon$, then either $e$ is a value or for any log $\ell$ there is an expression~$e'$ and log~$\ell'$ such that $e, \ell \to e', \ell'$.
\end{thm}

\begin{thm}[Preservation]
  If $\Gamma \vdash e : \tau \mid \varepsilon$, then for any $e'$ such that $e, \ell \to e', \ell'$, then $\Gamma \vdash e' : \tau \mid \varepsilon$.
\end{thm}

\noindent\textbf{N.B.} Here we're using the fact that we don't ever force purity.
After all, $\logn{\tilde{n}}$ steps to $\unit$, which is pure.
But we can always consider $\unit$ to be a program which may log data, but doesn't.

\begin{thm}[Effect Soundness]
  If $\Gamma \vdash e : \tau \mid \pureeff$ and $e, \ell \to^\ast e', \ell'$, then $\ell' = \ell$.
\end{thm}

\begin{thm}[Values are Pure]
  If $\Gamma \vdash v : \tau \mid \varepsilon$, then $\Gamma \vdash v : \tau \mid \pureeff$.
\end{thm}

\begin{itemize}
\item All of the theorems above are proven using techniques we've studied repeatedly this semester.
\end{itemize}

\section*{Categorical Semantics}
\label{sec:categ-semant}

\begin{itemize}
\item What are the categorical semantics for such a language?
\item Intuitively, it's kind of two languages in one: a pure language and an effectful language.
\item This is reflected in the categorical semantics: we have two categories, $\mathcal{P}$ for pure programs and $\mathcal{E}$ for effectful programs.
\item Each of these is cartesian closed: the product for both is the product type:
  \begin{itemize}
  \item The exponential for pure programs is $\tau_1 \xrightarrow{\pureeff} \tau_2$.
  \item The exponential for effectful programs is $\tau_1 \xrightarrow{\logeff} \tau_2$.
  \end{itemize}
\item However, there's more structure here: the fact that every pure program can be considered effectful is captured in the following lemma:
  \begin{lem}
    The following rule is admissible:
    $$\infer*[left=LiftEff]{\Gamma \vdash e : \tau \mid \pureeff}{\Gamma \vdash e : \tau \mid \logeff}$$
  \end{lem}
  This becomes a functor $F : \mathcal{P}  \to \mathcal{E}$.
\item We can also transform types and programs into the completely-pure world: an effectful program can be represented as a program which tracks the log as a list.
  We write $M\,\tau$ for the result of transforming a type as follows:
  $$
  M\,\tau = \left\{\begin{array}{ll}
      \utype \times \ltype{\mathbb{N}} & \tau = \utype\\
      \mathbb{N} \times \ltype{\mathbb{N}} & \tau = \mathbb{N}\\
      \ltype{\tau'} \times \ltype{\mathbb{N}} & \tau = \ltype{\tau'}\\
      (\tau_1 \times \tau_2) \times \ltype{\mathbb{N}} & \tau = \tau_1 \times \tau_2\\
      (\tau_1 + \tau_2) \times \ltype{\mathbb{N}} & \tau = \tau_1 + \tau_2\\
      (\tau_1 \xrightarrow{\pureeff} \tau_2) \times \ltype{\mathbb{N}} & \tau = \tau_1 \xrightarrow{\pureeff} \tau_2\\
      (\tau_1 \xrightarrow{\pureeff} (\tau_1 \times \ltype{\mathbb{N}})) \times \ltype{\mathbb{N}} & \tau = \tau_1 \xrightarrow{\logeff} \tau_2
    \end{array}\right.
  $$
\item Every effectful program $x : \tau \vdash e : \sigma \mid \logeff$ can be transformed into a program $\capture{e}$ such that $x : \tau \vdash e : M\,\sigma \mid \pureeff$.
  \begin{itemize}
  \item The resulting program collects all of its logged data in its output.
  \item Thus, $\hom[\mathcal{E}]{X}{Y} = \hom[\mathcal{P}]{X}{M\,Y}$
  \end{itemize}
\item Moreover, a program $x : \tau \vdash e : M\,\sigma \mid \varepsilon$ can be transformed into a program $x : M\,\tau \vdash \bind{e} : M\,\sigma \mid \varepsilon$.
  \begin{itemize}
  \item If $f : X \to Y$ and $g : Y \to Z$ in $\mathcal{E}$, then $\capture{f; g} = \capture{f}; \bind{\capture{g}}$
  \end{itemize}
\item Finally, there is an easy program $x : \tau \vdash (x, \nillist) : M\,\tau$ that plays well with bind.
\item Together, these make $M$ a monad
\end{itemize}

\begin{defn}[Monad]
  A \emph{monad} on a category $\mathcal{C}$ is an endofunctor $M : \mathcal{C} \to \mathcal{C}$ along with a natural transformation $\eta : \id[\mathcal{C}] \Rightarrow M$ and $\mu : M; M \Rightarrow M$ such that the following diagrams always commute:
  \begin{center}
    \begin{tikzcd}
      M\, X \ar[r, "\eta_X"] \ar[d,"M\,\eta_X"'] \ar[dr,dash,"{\id[X]}"] & M\, M\, X \ar[d, "\mu_X"] \\
      M\,M\,X \ar[r, "\mu_X"'] & M \, X
    \end{tikzcd}
  \end{center}
  \begin{center}
    \begin{tikzcd}
      M\,M\,M\,X \ar[r, "{\mu_{M\, X}}"] \ar[d, "{M\,\mu_X}"'] & M\, M\, X \ar[d, "\mu_X"] \\
      M\,M\,X \ar[r, "\mu_X"'] & M\, X
    \end{tikzcd}
  \end{center}
\end{defn}

\begin{defn}[Kleisli Category]
  For a category $\mathcal{C}$ and a monad $M : \mathcal{C} \to \mathcal{C}$, the \emph{Kleisli category}~$\mathcal{K}_M$ for $M$ has:
  \begin{itemize}
  \item \textbf{Objects}: Same as the objects of $\mathcal{C}$
  \item \textbf{Morphisms}: $\hom[\mathcal{K}_M]{X}{Y} = \hom[\mathcal{C}]{X}{M\,Y}$
  \item \textbf{Identities}: $\id[X] \in \hom[\mathcal{K}_M]{X}{X} = \eta_X \in \hom[\mathcal{C}]{X}{M\,X}$
  \item \textbf{Composition}: $f; g \in \hom[\mathcal{K}_M]{X}{Z}$ = $f; M\,g; \mu_Z \in \hom[\mathcal{C}]{X}{Z}$
  \end{itemize}
\end{defn}

\begin{itemize}
\item The fact that the Kleisli category is a lawful category is a homework problem.
\item Then, $\mathcal{E} = \mathcal{K}_M$ is the Kleisli category for the logging monad.
\end{itemize}

\end{document}

%%% Local Variables:
%%% mode: latex
%%% TeX-master: t
%%% TeX-engine: luatex
%%% End:
