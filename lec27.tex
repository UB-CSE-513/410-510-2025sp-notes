\documentclass{lecturenotes}

\usepackage[colorlinks,urlcolor=blue]{hyperref}
\usepackage{doi}
\usepackage{xspace}
\usepackage{fontspec}
\usepackage{enumerate}
\usepackage{mathpartir}
\usepackage{pl-syntax/pl-syntax}
\usepackage{forest}
\usepackage{stmaryrd}
\usepackage{epigraph}
\usepackage{xspace}

\setsansfont{Fira Code}

\newcommand{\real}{\ensuremath{\mathbb{R}}\xspace}
\newcommand{\complex}{\ensuremath{\mathbb{C}}\xspace}
\newcommand{\abs}[2]{\ensuremath{\lambda #1.\,#2}}
\newcommand{\tabs}[3]{\ensuremath{\lambda #1 \colon #2.\,#3}}
\newcommand{\dbabs}[1]{\ensuremath{\lambda.\,#1}}
\newcommand{\dbind}[1]{\ensuremath{\text{\textasciigrave}#1}}
\newcommand{\app}[2]{\ensuremath{#1\;#2}}
\newcommand{\utype}{\textsf{unit}\xspace}
\newcommand{\unit}{\ensuremath{\textsf{(}\mkern0.5mu\textsf{)}}}
\newcommand{\prodtype}[2]{\ensuremath{#1 \times #2}}
\newcommand{\pair}[2]{\ensuremath{(#1, #2)}}
\newcommand{\projl}[1]{\ensuremath{\pi_1\mkern2mu#1}}
\newcommand{\projr}[1]{\ensuremath{\pi_2\mkern3mu#1}}
\newcommand{\sumtype}[2]{\ensuremath{#1 + #2}}
\newcommand{\injl}[1]{\ensuremath{\textsf{inj}_1\mkern2mu#1}}
\newcommand{\injr}[1]{\ensuremath{\textsf{inj}_2\mkern3mu#1}}
\newcommand{\case}[5]{\ensuremath{\textsf{case}\mkern5mu#1\mkern5mu\textsf{of}\mkern5mu\injl{#2} \Rightarrow #3;\mkern5mu\injr{#4} \Rightarrow #5\mkern5mu\textsf{end}}}
\newcommand{\vtype}{\textsf{void}\xspace}
\newcommand{\vcase}[1]{\ensuremath{\textsf{case}\mkern5mu#1\mkern5mu\textsf{of}\mkern5mu\textsf{end}}}
\newcommand{\rectype}[2]{\ensuremath{\mu #1.\,#2}}
\newcommand{\roll}[1]{\textsf{roll}\mkern2mu#1}
\newcommand{\unroll}[1]{\textsf{unroll}\mkern2mu#1}
\newcommand{\fatype}[2]{\ensuremath{\forall #1.\,#2}}
\newcommand{\Abs}[2]{\Lambda #1.\,#2}
\newcommand{\App}[2]{#1\;[#2]}
\newcommand{\extype}[2]{\ensuremath{\exists #1.\,#2}}
\newcommand{\pack}[3]{\ensuremath{\textsf{pack}\mkern5mu#1\mathrel{\textsf{as}}#2\mathrel{\textsf{in}}#3}}
\newcommand{\unpack}[4]{\ensuremath{\textsf{unpack}\mkern5mu#1\mathrel{\textsf{as}} #2, #3 \mathrel{\textsf{in}} #4}}

\newcommand{\at}{\ensuremath{\mathrel{@}}}
\newcommand{\binval}[4]{\ensuremath{\mathcal{V}(#1 \at #2)(#3 \sim #4)}}
\newcommand{\binexpr}[4]{\ensuremath{\mathcal{E}(#1 \at #2)(#3 \sim #4)}}
\newcommand{\bintypectxt}[2]{\ensuremath{\mathcal{T}(#1)(#2)}}
\newcommand{\binctxt}[4]{\ensuremath{\mathcal{K}(#1 \at #2)(#3 \sim #4)}}

\newcommand{\FV}{\text{FV}}
\newcommand{\BV}{\text{BV}}

\newcommand{\toform}[1]{\ensuremath{\lceil #1 \rceil}}
\newcommand{\totype}[1]{\ensuremath{\lfloor #1 \rfloor}}

\newcommand{\neutral}[1]{#1\;\text{ne}}
\newcommand{\nf}[1]{#1\;\text{nf}}

\newcommand{\subtype}{\ensuremath{\mathrel{\mathord{<}\mathord{:}}}}

\title{Parametricity}
\coursenumber{CSE 410/510}
\coursename{Programming Language Theory}
\lecturenumber{27}
\semester{Spring 2025}
\professor{Professor Andrew K. Hirsch}

\begin{document}
\maketitle

\begin{itemize}
\item At the beginning of the lecture, we start with the paper \emph{Types, Abstraction and Parametric Polymorphism} by John C. Reynolds.
\item In this paper, Reynolds introduced the idea of \emph{parametricity} by using an example.
\item Two mathematics professors, Descartes and Bessel, define the concept of complex numbers \complex in different ways.
\item Even though they define the same concept differently, these definitions are interchangeable.
\item This is because both of them define the complex number in bijection function \textsf{f : \prodtype{\real}{\real}\ \rightarrow\ \complex}
      and operations(converting \real to \complex, addition, multiplication, etc.), so that \complex is now an abstraction of multiple sets.
\end{itemize}

\section{Parametricity}\label{sec:parametricity}

\begin{itemize}
\item Recall the identity function in CBV system F:
    \[
        \Abs{X}{\tabs{x}{X}{x}}\ :\ \fatype{X}{X \to X}
    \]
    Parametricity means that this function must \emph{treat its type parameters uniformly}.
\item A parametric polymorphic program is not allowed to do different things depending on the type of its argument.
\item Now, suppose we want to allow two variables or programs of polymorphic type to be related.
\item We should be able to express this by using binary logical relations:
    \[
        \mathcal{V}(\tau \at \rho )(v_1 \sim v_2)
    \]
    Here, we map the type to the binary relation on the values. This means that ``$v_1$ and $v_2$ are indistinguishable,
    where free type variables in $\tau$ are interpreted as $\rho$''. 
\end{itemize}

\section{The Binary Relation}\label{sec:the-binary-relation}
\begin{itemize}
    \item Again, we use $\mathcal{V}(\tau)(v_1 \sim v_2)$ for the value relation, and $\mathcal{E}(\tau)(e_1 \sim e_2)$ for the expression relation.
    \item Recall $\tau ::= X\ \vert\ \tau_1 \to \tau_2\ \vert\ \fatype{X}{\tau}$.
\end{itemize}

\[
\begin{array}{r@{\;\triangleq\;}l}
    \binval{X}{\rho}{v_1}{v_2} & \rho(X)(v_1, v_2)\\
    \binval{\tau_1 \to \tau_2}{\rho}{v_1}{v_2} &
        \exists x, e_1, y, e_2.\, \hspace{-1.5em}\begin{array}[t]{l}\hspace{1.15em}v_1 = \tabs{x}{\tau_1}{e_1}\\{} \land v_2 = \tabs{y}{\tau_2}{e_2}\\{} \land (\forall a_1, a_2.\, \binval{\tau_1}{\rho}{a_1}{a_2}) \Rightarrow \binexpr{\tau_2}{\rho}{e_1[x \mapsto a_1]}{e_2[y \mapsto a_2]}\end{array}\\
    \binval{\fatype{X}{\tau}}{\rho}{v_1}{v_2} &
        \exists e_1, e_2.\, \hspace{-1.5em}\begin{array}[t]{l}\hspace{1.15em}v_1 = \Abs{X}{e_1}\\{} \land v_2 = \Abs{X}{e_2}\\ {} \land \forall R \subseteq \text{Val} \times \text{Val}.\, \binexpr{\tau}{\rho[X \mapsto R]}{e_1}{e_2}\end{array}\\

    \bintypectxt{\Delta}{\rho} & \Delta \subseteq \text{dom}(\rho) \\
    \binctxt{\Gamma}{\rho}{\gamma_1}{\gamma_2} & \forall x : \tau \in \Gamma.\, \binval{\tau}{\rho}{\gamma_1(x)}{\gamma_2(x)} \\
    \Delta; \Gamma \vDash e_1 \sim e_2 : \tau & \forall \rho.\, \bintypectxt{\Delta}{\rho} \Rightarrow \forall \gamma_1, \gamma_2.\, \binctxt{\Gamma}{\rho}{\gamma_1}{\gamma_2} \Rightarrow \binexpr{\tau}{\rho}{e_1[\gamma_1]}{e_2[\gamma_2]}    
\end{array}
\]

\begin{itemize}
    \item For relation of expression $\fatype{X}{\tau}$, the type input of both $e_1$ and $e_2$ must be the same as $X$.
    \item For all relations $R$, $e_1$ equivalent to $e_2$ no matter how we interpret $X$.
    \item The context rule $\mathcal{K}$ is quite similar to the context rule we have seen before.
    \item Now we can finally define the parametricity theorem and prove it!
\end{itemize}

\section{Parametricity Theorem}\label{thm:parametricity-theorem}
\begin{thm}[Fundamental Theorem of Parametricity]
    $\Delta; \Gamma \vdash e : \tau \implies \Delta; \Gamma \vDash  e \sim e : \tau$.
\end{thm}
\begin{proof}
    We will only prove $\forall$-I and $\forall$-E cases here:

    \noindent\textbf{Case \forall-I:}\\
    In this case, $e = \Abs{X}{e'}$ and $\tau = \fatype{X}{\tau'}$.
    Moreover, we have $\Delta; \Gamma \vdash e' : \tau'$. We pick $\rho$ such that $\bintypectxt{\Delta}{\rho}$
    and $\gamma_1, \gamma_2$ such that $\binctxt{\Gamma}{\rho}{\gamma_1}{\gamma_2}$. Let $R$ be the relation, we claim a lemma:
    \begin{lem}
        $\forall R, \binctxt{\Gamma}{\rho[X \mapsto R]}{e'[\gamma_1]}{e'[\gamma_2]}$.
    \end{lem}
    \noindent$\binexpr{\tau'}{\rho[X \mapsto R]}{e'[\gamma_1]}{e'[\gamma_2]}$ holds by induction hypothesis.
    Since Lemma 1 holds for all $R$, we have $\binval{\fatype{x}{\tau'}}{R}{\Abs{X}{e'[\gamma_1]}}{\Abs{X}{e'[\gamma_2]}}$.\\

    \noindent\textbf{Case \forall-E:}\\
    In this case, We have $e = e'[\tau_1]$, $\tau = \tau_2[X \mapsto \tau_1]$ and $\Delta; \Gamma \vdash e_1 : \fatype{X}{\tau_2}$.
    We pick $\rho, \gamma_1, \gamma_2$ be the same as in the previous case. By induction hypothesis, we have
    $\binexpr{\fatype{X}{\tau_2}}{\rho}{e'[\gamma_1]}{e'[\gamma_2]}$. Thus, $e'[\gamma_1] \to^\ast \Abs{X}{e_1}$, $e'[\gamma_2] \to^\ast \Abs{X}{e_2}$.
    We choose to interpret $X$ as $\mathcal{V}(\tau_1 \at \rho)$, so $\binexpr{\tau_2}{\rho[X \mapsto \mathcal{V}(\tau_1 \at \rho)]}{e_1}{e_2}$
    We claim another lemma:
    \begin{lem}
        $\binexpr{\tau_2[X \mapsto \mathcal{V}(\tau_1 \at \rho)]}{\rho}{e_1[X \mapsto \gamma_1]}{e_2[X \mapsto \gamma_2]}$.
    \end{lem}
    \noindent$\App{(e[\gamma_1])}{\tau_1} \to^\ast \App{(\Abs{X}{e_1})}{\tau_1} \to e_1[X \mapsto \gamma_1]$.
\end{proof}

\section{Free Theorem}\label{sec:free-theorem}
\begin{itemize}
    \item A free theorem is a theorem that states that if a polymorphic function is defined in a certain way, then it must satisfy certain properties.
    \item We can get the meaning of the behavior of the polymorphic function by just looking at the type
          without any knowledge of the implementation --- which is ``free''.
    \item For details of the free theorem, please refer to the paper \emph{Theorems for Free} by Philip Wadler.
\end{itemize}
\begin{thm}[Free Theorem]
    There is no e such that $\cdot; \cdot \vdash e : \fatype{X}{X}$.
\end{thm}
\begin{proof}
    By fundamental theorem, $\binexpr{\fatype{X}{X}}{\rho}{e}{e}$.
    $e \to^\ast \Abs{X}{e'}$. For any $R$, $\binexpr{X}{\rho[X \mapsto R]}{e'}{e'}$.
    We pick $R=\emptyset$, then $e' \to^\ast v, \binval{X}{[X \mapsto R]}{v}{v}$.
    This means $R(v,v)$, but $R=\emptyset$, which is a contradiction.
    Therefore, there is no such $e$.
\end{proof}
\end{document}

%%% Local Variables:
%%% mode: latex
%%% TeX-master: t
%%% TeX-engine: luatex
%%% End: