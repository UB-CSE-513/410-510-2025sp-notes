\documentclass{lecturenotes}

\usepackage[colorlinks,urlcolor=blue]{hyperref}
\usepackage{doi}
\usepackage{xspace}
\usepackage{fontspec}
\usepackage{enumerate}
\usepackage{mathpartir}
\usepackage{pl-syntax/pl-syntax}
\usepackage{forest}
\usepackage{stmaryrd}
\usepackage{epigraph}
\usepackage{xspace}

\setsansfont{Fira Code}

\newcommand{\abs}[2]{\ensuremath{\lambda #1.\,#2}}
\newcommand{\tabs}[3]{\ensuremath{\lambda #1 \colon #2.\,#3}}
\newcommand{\dbabs}[1]{\ensuremath{\lambda.\,#1}}
\newcommand{\dbind}[1]{\ensuremath{\text{\textasciigrave}#1}}
\newcommand{\app}[2]{\ensuremath{#1\;#2}}
\newcommand{\utype}{\textsf{unit}\xspace}
\newcommand{\unit}{\ensuremath{\textsf{(}\mkern0.5mu\textsf{)}}}
\newcommand{\prodtype}[2]{\ensuremath{#1 \times #2}}
\newcommand{\pair}[2]{\ensuremath{(#1, #2)}}
\newcommand{\projl}[1]{\ensuremath{\pi_1\mkern2mu#1}}
\newcommand{\projr}[1]{\ensuremath{\pi_2\mkern3mu#1}}
\newcommand{\sumtype}[2]{\ensuremath{#1 + #2}}
\newcommand{\injl}[1]{\ensuremath{\textsf{inj}_1\mkern2mu#1}}
\newcommand{\injr}[1]{\ensuremath{\textsf{inj}_2\mkern3mu#1}}
\newcommand{\case}[5]{\ensuremath{\textsf{case}\mkern5mu#1\mkern5mu\textsf{of}\mkern5mu\injl{#2} \Rightarrow #3;\mkern5mu\injr{#4} \Rightarrow #5\mkern5mu\textsf{end}}}
\newcommand{\vtype}{\textsf{void}\xspace}
\newcommand{\vcase}[1]{\ensuremath{\textsf{case}\mkern5mu#1\mkern5mu\textsf{of}\mkern5mu\textsf{end}}}

\newcommand{\FV}{\text{FV}}
\newcommand{\BV}{\text{BV}}

\newcommand{\toform}[1]{\ensuremath{\lceil #1 \rceil}}
\newcommand{\totype}[1]{\ensuremath{\lfloor #1 \rfloor}}

\newcommand{\neutral}[1]{#1\;\text{ne}}
\newcommand{\nf}[1]{#1\;\text{nf}}

\title{The Propisitions-as-Types Principle}
\coursenumber{CSE 410/510}
\coursename{Programming Language Theory}
\lecturenumber{18}
\semester{Spring 2025}
\professor{Professor Andrew K. Hirsch}

\begin{document}
\maketitle

\begin{itemize}
\item We've been looking at types in $\lambda$~calculus so far, and in particular at the simple type constructors $\times$, $+$, \utype, and \vtype.
\item In Agda, we used these type constructors for logic. 
\item In this lecture, we will see that there is in fact a deep connection between intuitionistic propositional logic and the type system we have seen so far.
\item In order to see this, we first have to define intuitionistic propositional logic and its proof theory---i.e., what are the logical statements and how do we prove them?
\end{itemize}

\section{The Language of Intuitionistic Propositional Formulae}
\label{sec:lang-intu-prop}

\begin{syntax}
  \abstractCategory[Basic Propositions]{p,q,\dots}
  \category[Formulae]{\varphi, \psi, \dots} \alternative{\top} \alternative{\bot} \alternative{p}\\
  \alternative{\varphi \land \psi} \alternative{\varphi \lor \psi} \alternative{\varphi \to \psi} 
\end{syntax}

\begin{itemize}
\item The formulae, or logical statements, of intuitionistic propositional logic can be found above.
\item Here, the basic propositions $p, q, \dots$ represent statements that cannot be broken down logically.
  Usually, these are statements about the world, like ``It is snowing in Buffalo'' or ``The value of $x$ in the current state is $510$''.
\item We also have the ``basic'' propositions $\top$ (which is always true) and $\bot$ (which is always false).
  \begin{itemize}
  \item So why don't we just insist that they are in the set of basic propositions?
  \item We could, but it's more natural to put them directly in the syntax of our language.
  \item Because of that, it's more natural to view them as constants or ``$0$-ary connectives''.
  \end{itemize}
\item We can then combine these with the three binary connectives $\varphi \land \psi$ (``$\varphi$ \emph{and} $\psi$''), $\varphi \lor \psi$ (``$\varphi$ \emph{or} $\psi$), and $\varphi \to \psi$ (``$\varphi$ \emph{implies} $\psi$'').
\item As in Agda, we use $\lnot \varphi$ as shorthand for $\varphi \to \bot$ and read it as ``not $\varphi$''.
  We can think of $\lnot$ as a unary connective.
\item At this point, you should have some intuition for how these connectives work, both from Agda and from your previous experience with logic.
\item However, we can do better: we can formally define what proofs of these formulae look like.
\end{itemize}

\section{Proof Theory}
\label{sec:proof-theory}

\begin{itemize}
\item Today, we will be studying a form of proof theory called \emph{natural deduction with localized hypotheses}.
  \begin{itemize}
  \item For those of you in the know, this is sort of in between natural deduction and sequent calculus.
  \item It keeps track of its assumptions in a context like sequent calculus, but is otherwise the same as natural deduction.
  \item This allows us to get rid of ``justified'' versus ``unjustified'' assumptions, which causes a lot of confusion in natural deduction proofs.
  \end{itemize}
\item Proof theory is defined as a relation $\Gamma \vdash \varphi$ where $\Gamma$ is a set of formulae and $\varphi$ is a formula.
  We read this relation as ``if everything in $\Gamma$ is true, then $\varphi$ is true''.
\item This looks very similar to the notation for type systems.
  In fact, the notation for type systems is inspired by this notation!
  \begin{itemize}
  \item Historically, this notation comes from Gentzen, who invented natural deduction and sequent calculus.
  \item However, he used it for sequent calculus; natural deduction has a one-place relation $\vdash \varphi$.
  \item This notation comes from Frege, one of the first people to study formal proof.
  \item Frege also invented the two-dimensional ``fraction-like'' syntax for defining and proving inductive relations, though he  combined it with the turnstile ($\vdash$) notation to create some bizarre contortions of notation.
  \item Gentzen refined it to its modern version.
  \end{itemize}
\item We can write the proof system for intuitionsitic propositional logic as follows:
\end{itemize}

\begin{mathparpagebreakable}
  \infer*[left=Axiom]{\varphi \in \Gamma}{\Gamma \vdash \varphi}\\
  \infer*[left=$\top$-I]{ }{\Gamma \vdash \top}\and
  \infer*[left=$\bot$-E]{\Gamma \vdash \bot}{\Gamma \vdash \varphi}\\
  \infer*[left=$\land$-I]{\Gamma \vdash \varphi\\ \Gamma \vdash \psi}{\Gamma \vdash \varphi \land \psi}\and
  \infer*[left=$\land$-E$_1$]{\Gamma \vdash \varphi \land \psi}{\Gamma \vdash \varphi}\and
  \infer*[left=$\land$-E$_2$]{\Gamma \vdash \varphi \land \psi}{\Gamma \vdash \psi}\\
  \infer*[left=$\lor$-I$_1$]{\Gamma \vdash \varphi}{\Gamma \vdash \varphi \lor \psi}\and
  \infer*[left=$\lor$-I$_2$]{\Gamma \vdash \psi}{\Gamma \vdash \varphi \lor \psi}\and
  \infer*[left=$\lor$-E]{\Gamma \vdash \varphi \lor \psi\\\Gamma, \varphi \vdash \chi\\\Gamma, \psi \vdash \chi}{\Gamma \vdash \chi}\\
  \infer*[left=$\to$-I]{\Gamma, \varphi \vdash \psi}{\Gamma \vdash \varphi \to \psi} \and
  \infer*[left=$\to$-E]{\Gamma \vdash \varphi \to \psi\\ \Gamma \vdash \varphi}{\Gamma \vdash \psi}
\end{mathparpagebreakable}
\begin{itemize}
\item If $\varphi \in \Gamma$, then $\varphi$ is assumed to be true and so we can always prove it.
\item Similarly, we can always prove $\top$.
\item However, we can never prove $\bot$---if we do, then something has gone wrong and the universe blows up.
\item To prove $\varphi \land \psi$, you must prove $\varphi$ and $\psi$.
  Given such a proof, you can extract either a proof of $\varphi$ or a proof of $\psi$ from it.
\item To prove $\varphi \lor \psi$, you must prove either $\varphi$ or $\psi$.
  Given such a proof, you can prove $\chi$ if you can prove $\chi$ no matter which of $\varphi$ or $\psi$ was used to prove $\varphi \lor\psi$.
\item To prove $\varphi \to \psi$, assume $\varphi$ is true and prove $\psi$.
  Given such a proof, if $\varphi$ is true you can prove $\psi$.
\item You might ask about the basic propositions.
  Without making an assumption, we have no idea if they are true or false---after all, this system doesn't know anything about whether it's raining in Buffalo or not!
  Therefore, we can only prove anything about them using the \textsc{Axiom} rule.
\end{itemize}

\newpage
\section{Propositions as Types}
\label{sec:prop-as-types}

\begin{itemize}
\item Hopefully, the connection between this system in the previous system is obvious: we've simply deleted the programs and renamed some constructors.
  \begin{itemize}
  \item Well, we also started using Greek more, but that's not in the system itself.
  \end{itemize}
\item We can formalize this via a translation.
  For a type~$\tau$, we'll write $\toform{\tau}$ for the formula that corresponds to $\tau$, and given a formula~$\varphi$, we'll write $\totype{\tau}$ for the type corresponding to $\varphi$.
\end{itemize}

$$\toform{\tau} = \left\{\begin{array}{ll}
  \top & \text{if}\;\tau = \utype\\
  \bot & \text{if}\;\tau = \vtype\\
  \toform{\tau_1} \land \toform{\tau_2} & \text{if}\;\tau=\tau_1 \times \tau_2\\
  \toform{\tau_1} \lor \toform{\tau_2} & \text{if}\;\tau=\tau_1 + \tau_2\\
  \toform{\tau_1} \to \toform{\tau_2} & \text{if}\;\tau=\tau_1 \to \tau_2
\end{array}\right.
\hspace{5em}
\totype{\varphi} = \left\{\begin{array}{ll}
  \utype & \text{if}\;\varphi = \top\\
  \vtype & \text{if}\;\varphi = \bot\\
  \totype{\psi} \times \totype{\chi} & \text{if}\;\varphi = \psi \land \chi\\
  \totype{\psi} + \totype{\chi} & \text{if}\;\varphi = \psi \lor \chi\\
  \totype{\psi} \to \totype{\chi} & \text{if}\;\varphi = \psi \to \chi
\end{array}\right.
$$

\begin{itemize}
\item Two observations made without proof:
  \begin{itemize}
  \item $\toform{-}$ and $\totype{-}$ form an isomorphism.
    In other words, $\toform{\totype{\varphi}} = \varphi$ and $\totype{\toform{\tau}} = \tau$ for every formula~$\varphi$ and type~$\tau$.
  \item Any typing derivation $\Gamma \vdash e : \tau$ in $\lambda$~calculus corresponds to exactly one proof of $\toform{\Gamma} \vdash \toform{\tau}$.
    Moreover, for any proof $\Gamma\vdash \varphi$ we can find a program $e$ such that $\totype{\Gamma} \vdash e : \totype{\varphi}$.
    This $e$ is unique up to the choice of names.
  \end{itemize}
\item Together, these form the \emph{Curry-Howard Correspondence} for intuitionistic propositional logic.
\item The Curry-Howard correspondence comes from a simple principle of logic:
  \begin{itemize}
  \item \textbf{\large Propositions are Types}
  \item \textbf{\large Programs are Proofs}
  \end{itemize}
\item We call this principle the \emph{propositions-as-types} principle.
\item Later, we will expand this to the principle of \emph{computational trinitarianism}.
\item Note that most people use the phrases ``Curry-Howard Correspondence'' and ``Propositions-as-Types Principle'' interchangeably.
  You will also hear ``Curry-Howard Isomorphism'', even though an isomorphism suggests some structure being preserved that is ill-defined.
\end{itemize}

\section{Normal Forms and Consistency}
\label{sec:norm-forms-cons}

\begin{itemize}
\item Remember that a $\lambda$~term is in $\beta$-normal form if it cannot $\beta$ reduce any further.
\item More specifically, a $\lambda$~term is in $\beta$-normal form if it is a variable, stuck on a variable, stuck on something stuck on a variable, or a function whose body is in normal form.
\item We can describe this mathematically by defining normal forms mutually inductively with \emph{neutral} forms, which are stuck on some variable.
\item We write $\neutral{e}$ to mean that $e$ is in neutral form, and $\nf{e}$ to mean $e$ is in normal form.
\end{itemize}

\begin{mathpar}
  \infer{ }{\neutral{x}} \and
  \infer{\neutral{e}}{\neutral{\projl{e}}}\and
  \infer{\neutral{e}}{\neutral{\projr{e}}} \\
  \infer{\neutral{e_1}\\ \nf{e_2}\\ \nf{e_3}}{\neutral{\case{e_1}{x}{e_2}{y}{e_3}}}\\
  \infer{\neutral{e_1}}{\neutral{\vcase{e_1}}}\\
  \infer{\neutral{e}}{\nf{e}}\\
  \infer{ }{\nf{\unit}}\and
  \infer{\nf{e}}{\nf{\abs{x}{e}}}\\
  \infer{\nf{e_1}\\\nf{e_2}}{\nf{\pair{e_1}{e_2}}}\and
  \infer{\nf{e}}{\nf{\injl{e}}}\and
  \infer{\nf{e}}{\nf{\injr{e}}}
\end{mathpar}

\begin{itemize}
\item Think about what a typing derivation for a normal form looks like: first, we use all of the elimination rules, then we move into ``introduction mode'' and we are no longer allowed to use elimination rules, only introduction rules.
\item We say that a proof is in normal form if and only if the $\lambda$~term corresponding to it is in normal form.
\item We then have two easy lemmata:
\end{itemize}

\begin{lem}
  There are no closed neutral-form expressions.
  In other words, if $v$ is closed, then $\lnot (\neutral{v})$ 
\end{lem}
\begin{proof}
  A simple induction on the neutral-form rules.
  Note that the only ``top-level'' neutral-form rule that does not require a subterm to be neutral is the variable rule.
\end{proof}

\begin{lem}
  There is no normal-form term of type $\vtype$.
  In other words $\not\vdash v : \vtype$ when $\nf{v}$.
\end{lem}
\begin{proof}
  None of the introduction forms can create something of type $\vtype$, and there are no closed neutral forms.
\end{proof}

\begin{cor}
  There is no normal-form proof of falsity.
  In other words, if $\vdash \bot$ then the proof is not in normal form.
\end{cor}

\begin{itemize}
\item This gets us a good way towards \emph{consistency}: it is impossible to prove falsity in intuitionistic propositional logic.
\item However, we only know this for normal-form proofs.
\item Maybe falsity is provable, but only for some not-normal-form proof?
\item This turns out to be impossible!
\end{itemize}

\begin{thm}[Normalization]
  If $\Gamma \vdash \varphi$, then there is a normal-form proof of $\Gamma \vdash \varphi$.
\end{thm}
\begin{proof}[Proof Sketch]
  Since $\Gamma \vdash \varphi$, there is an (unique) $e$ such that $\totype{\Gamma} \vdash e : \totype{\varphi}$.
  We can start $\beta$ reducing $e$ until it will not $\beta$ reduce any more.
  By preservation, this will still have the right type.
  
  But will we ever finish?
  It turns out the answer is \emph{yes}, but we don't have the technology to prove it yet.
\end{proof}

\begin{thm}[Consistency]
  $\not\vdash \bot$
\end{thm}
\begin{proof}
  If $\vdash \bot$, then by normalization, there would be a normal-form proof of falsity.
  But we already said that this is impossible.
\end{proof}

\end{document}

%%% Local Variables:
%%% mode: latex
%%% TeX-master: t
%%% TeX-engine: luatex
%%% End:
