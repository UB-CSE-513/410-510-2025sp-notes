 
\documentclass{lecturenotes}

\usepackage[colorlinks,urlcolor=blue]{hyperref}
\usepackage{doi}
\usepackage{xspace}
\usepackage{agda}
\usepackage{fontspec}
\usepackage{enumerate}
\usepackage{mathpartir}
\setsansfont{Fira Code}
\usepackage{newunicodechar}
\newunicodechar{∣}{\ensuremath{\mid}}
\newunicodechar{′}{\ensuremath{{}^\prime}}
\newunicodechar{ˡ}{\ensuremath{{}^{\textsf{l}}}}
\newunicodechar{ʳ}{\ensuremath{{}^{\textsf{r}}}}
\newunicodechar{≤}{\ensuremath{\mathord{\leq}}}
\newunicodechar{≡}{\ensuremath{\mathord{\equiv}}}
\newunicodechar{≐}{\ensuremath{\mathord{\doteq}}}
\newunicodechar{∘}{\ensuremath{\circ}}
\newunicodechar{≃}{\ensuremath{\simeq}}
\newunicodechar{≲}{\ensuremath{\precsim}}

\newcommand{\agdanats}{\textsf{ℕ}\xspace}

\newcommand{\nil}{\ensuremath{\textsf{[]}}}
\newcommand{\cons}{\ensuremath{\mathbin{\textsf{::}}}}
\newcommand{\app}{\ensuremath{\mathbin{\textsf{++}}}}

\title{Connectives}
\coursenumber{CSE 410/510}
\coursename{Programming Language Theory}
\lecturenumber{6}
\semester{Spring 2025}
\professor{Professor Andrew K. Hirsch}

\begin{document}
\maketitle


After many kinds of induction seen in the previous lectures, it is the time for logic in a very formal way.
Logic is the basis of proof, and so far we have been proving things, in fact. How do proofs really work in the background?
Much must be covered to understand this. Today we will talk of conjunctions and other logical connectives, of acidity two or zero.
These connective appear somewhat in natural language, like with terms as "and" or "or".
We will study these connectives without talking about semantics and truth, mind that, as it is usually done with truth tables and similar structures. 
Let us start with the logical syntax. The language we will employ comprises atomic propositions like φ (phi) and ψ (psy), which can be further composed with the connectives.
In order for g-for a formula "to hold", or for it "to be true" , we need evidence of that.
The various logical connectives will be defined given the combinations of their being true or not in some metalanguage, as seen here:
If φ  holds and ψ hods, then phi AND spy also holds
If $\phi$ holds, then phi OR spy is also holds
Phi then Psy holds, despite in the case where phi holds and pst does not hold.
T (True) is a proposition that always holds, anddoes that trivially.
$\bot$ (False) is a proposition that never holds.
In this class we will try not to talk about truth and the metaphysical assumptions related to it, but we will prefer to talk about the evidence of a proposition.
The zero-place connective T, for example, is a formula you always have evidence for, instead of being the formula that is always true.
We might add that for everything your have evidence, that is true, but we leave this discussion to the epistemologists.
In Agda, we define conjunction in this way, with the symbol of "times":

\begin{code}
record _×_ (A B : Set) : Set where
  constructor ⟨_,_⟩
  field
    proj₁ : A
    proj₂ : B
open _×_
\end{code}

Now we can introduce the conjunction in an infix notation in this example, in which we give proof of the conjunction.
To do this we add in the holes Agda provides us with the poof of the LHS and that of the RHS, which are the arguments of the conjunction function.
We do not need to write the conjunction again in the goals of Agda.
Let's try out our new connective with a simple anonymous program, to see how it works: 

\begin{code}
_: (1 ≡ 1)  x  (2 ≡ 1 + 1)
_ = ⟨ refl , refl ⟩
\end{code}

By definition of conjunction as a data type, which is alternative to what we have just seen, we would have the following constructors.
Mind the upper commas after the symbol of each type:

\begin{code}
data _×′_ (A B : Set) : Set where

  ⟨_,_⟩′ :
      A
    → B
      -----
    → A ×′ B

proj₁′ : ∀ {A B : Set}
  → A ×′ B
    -----
  → A
proj₁′ ⟨ x , y ⟩′ = x

proj₂′ : ∀ {A B : Set}
  → A ×′ B
    -----
  → B
proj₂′ ⟨ x , y ⟩′ = y
\end{code}

The two definitions of conjunctions as the function `_×_` and as data type `_×′_` capture the same concept, but we need to bridge them with specific laws, called eta-laws:

\begin{code}
η-×′ : ∀ {A B : Set} (w : A ×′ B) → ⟨ proj₁′ w , proj₂′ w ⟩′ ≡ w
η-×′ ⟨ x , y ⟩′ = refl
\end{code}

As a note, we used the lambda function because Agda prefers to do pattern matching with some function.
Eta-laws have a similar form and they are even easier to prove for Agda.
The evidence of A and the evidence of B are enough to prove the `A ×′ B`. Nothing else is asked more.
Why do we call conjunction as a product? In his days, Boole was playing with a similar idea, that we can explain here.
Consider the Boolean types and the Tri type below, which we combine in all possible ordered pairs of (Bool, Tri):

\begin{code}
data Bool : Set where
  true  : Bool
  false : Bool

data Tri : Set where
  aa : Tri
  bb : Tri
  cc : Tri
\end{code}

And instead of counting all the cases by hand, we can have a function that does that for us automatically, resulting in six cases of type `Bool × Tri`.

\begin{code}
×-count : Bool × Tri → ℕ
×-count ⟨ true  , aa ⟩  =  1
×-count ⟨ true  , bb ⟩  =  2
×-count ⟨ true  , cc ⟩  =  3
×-count ⟨ false , aa ⟩  =  4
×-count ⟨ false , bb ⟩  =  5
×-count ⟨ false , cc ⟩  =  6
\end{code}

All these types are finite, that is certain. Let us commute them.
`Bool × Tri` is not the same as `Tri × Bool`, but they are isomorphic. Indeed, the right way to think of equality between types is with isomorphism.
Let's move to a zero-place connective, truth, also called Top. Truth is trivial, we do not need to provide proof of it.
In fact, the eta-law for truth says no further expansion is needed.

\begin{code}
record ⊤ : Set where
constructor tt
\end{code}

The eta-laws for True are spelled out in this way:

\begin{code}
η-⊤ : ∀ (w : ⊤) → tt ≡ w
η-⊤ w = refl
\end{code}

Moreover, the truth connective makes conjunction collapse to identity.

\begin{code}
⊤-identityˡ : ∀ {A : Set} → ⊤ × A ≃ A
⊤-identityˡ =
  record
    { to      = λ{ ⟨ tt , x ⟩ → x }
    ; from    = λ{ x → ⟨ tt , x ⟩ }
    ; from∘to = λ{ ⟨ tt , x ⟩ → refl }
    ; to∘from = λ{ x → refl }
    }
\end{code}

Now it is time for disjunction, a two-place connective spelled out as a weird-looking plus with a U on the outside.
It is a datatype with two constructors, inj1 and inj2. 

\begin{code}
data _⊎_ (A B : Set) : Set where
  inj₁ :  A  → A ⊎ B
  inj₂ : B → A ⊎ B
\end{code}

On finite sets, this datatype actually sums, so the symbol is appropriate.
We see this as we did for conjunction, counting the types and their disjunction cases.
In the case of falsity, also called Bottom, we do not have a constructor, but just a datatype.
Why so? Because you cannot give evidence for falsity, it is impossible.
Actually, given the logical rule of explosion, if you find the evidence of falsity, everything goes, you can derive whatever statement.
In other words, if you are trying to prove something and you hit bottom, there is no reason to carry on with the statement you are considering.

\begin{code}
data ⊥ : Set where
  -- no clauses!
\end{code}

There is no introduction rule for False, which would be very eerie, but there is one for elimination, as we mentioned.
If we reached False, we can derive any statement we want, and not by chance this is also known as the rule of explosion,
Another name of it is `ex falso quodlibe', from Latin, since tje law  was discovered in the Middle Ages.

\begin{code}
⊥-elim : ∀ {A : Set}  → ⊥  → A
⊥-elim ()
\end{code}

Moreover, it is easy to prove that is you have two proofs of Bottom, they must be equal.

\begin{code}
uniq-⊥ : ∀ {C : Set} (h : ⊥ → C) (w : ⊥) → ⊥-elim w ≡ h w
uniq-⊥ h ()
\end{code}

Finally, the implication connective is just a function. Given that you have evidence of the antecedent, then you have evidence of the consequent. 

\begin{code}
→-elim : ∀ {A B : Set}
  → (A → B)
  → A
    -------
  → B
→-elim L M = L M
\end{code}

As we have learnt in baby logic classes with truth tables, there are three cases in which the implication is true, and only one in which it is false.
This totals to nine possible combinations with the Tri type, or 3 squared.

\begin{code}
→-count : (Bool → Tri) → ℕ
→-count f with f true | f false
...          | aa     | aa      =   1
...          | aa     | bb      =   2
...          | aa     | cc      =   3
...          | bb     | aa      =   4
...          | bb     | bb      =   5
...          | bb     | cc      =   6
...          | cc     | aa      =   7
...          | cc     | bb      =   8
...          | cc     | cc      =   9
\end{code}

This is just a power law, we understand, and it is nothing more than currying in functional programming. 
A note on the `with` clause appearing in the last code block.
That special term is used for adding something to the pattern matching, and helps use reusing previous statements, like helper functions.
It also allows to use the three dots at the beginning of each case of the matching, to avoid repeating the same thing over and over.
A final mentions is reserved to the fact that conjunction and disjunction distribute over each other, which is spelled out like this in Agda: 

\begin{code}
×-distrib-⊎ : ∀ {A B C : Set} → (A ⊎ B) × C ≃ (A × C) ⊎ (B × C)
×-distrib-⊎ =
  record
    { to      = λ{ ⟨ inj₁ x , z ⟩ → (inj₁ ⟨ x , z ⟩)
                 ; ⟨ inj₂ y , z ⟩ → (inj₂ ⟨ y , z ⟩)
                 }
    ; from    = λ{ (inj₁ ⟨ x , z ⟩) → ⟨ inj₁ x , z ⟩
                 ; (inj₂ ⟨ y , z ⟩) → ⟨ inj₂ y , z ⟩
                 }
    ; from∘to = λ{ ⟨ inj₁ x , z ⟩ → refl
                 ; ⟨ inj₂ y , z ⟩ → refl
                 }
    ; to∘from = λ{ (inj₁ ⟨ x , z ⟩) → refl
                 ; (inj₂ ⟨ y , z ⟩) → refl
                 }
    }
\end{code}



 

\end{document}

%%% Local Variables:
%%% mode: latex
%%% TeX-master: t
%%% TeX-engine: luatex
%%% End:
