\documentclass{lecturenotes}

\usepackage[colorlinks,urlcolor=blue]{hyperref}
\usepackage{doi}
\usepackage{xspace}
\usepackage{fontspec}
\usepackage{enumerate}
\usepackage{mathpartir}
\setsansfont{Fira Code}
\usepackage{newunicodechar}
\newunicodechar{∣}{\ensuremath{\mid}}
\usepackage{stmaryrd}

\newunicodechar{∣}{\ensuremath{\mid}}
\newunicodechar{′}{\ensuremath{{}^\prime}}
\newunicodechar{ˡ}{\ensuremath{{}^{\textsf{l}}}}
\newunicodechar{ʳ}{\ensuremath{{}^{\textsf{r}}}}
\newunicodechar{ᵣ}{\ensuremath{{}_{\textsf{r}}}}
\newunicodechar{⊗}{\ensuremath{\otimes}}
\newunicodechar{∷}{\ensuremath{\mathrel{::}}}
\newunicodechar{∀}{\ensuremath{\textsf{\forall}}}
\newunicodechar{∎}{\ensuremath{\textsf{\blacksquare}}}
\newunicodechar{ℕ}{\ensuremath{\mathbb{N}}}
\newunicodechar{≤}{\ensuremath{\textsf{\le}}}
\newcommand{\subtype}{\ensuremath{\mathrel{\mathord{<}\mathord{:}}}}

\title{Subtyping}
\coursenumber{CSE 410/510}
\coursename{Programming Language Theory}
\lecturenumber{20}
\semester{Spring 2025}
\professor{Professor Andrew K. Hirsch}

\begin{document}
\maketitle

\section{System for Records}
Labels $\ell$ \newline
Example: $\{x=0; \; y=1\}$ $\{x:\mathbb{N}; \; y:\mathbb{N}\}$ \newline \newline
Types $\tau$ $\mathrel{::}$= $unit$ $\vert$ $\tau_1 \rightarrow \tau_2$ $\vert$
$\tau_1 \times \tau_2$ $\vert$ $\tau_1 + \tau_2$ $\vert$ 
$\{ \ell_1 : \tau_1 ; \; \dots ; \; \ell_n : \tau_n \}$ \newline \newline
Expressions $e$ $\mathrel{::}$= $\dots$ $\vert$ $\{\ell_1 = e_1; \; \dots ; \; \ell_n = e_n \}$ 
$\vert$ $e.\ell$ \newline \newline
$\{ \ell_1 = e_1 ; \dots ; \ell_n = e_n \} .\ell_i \rightarrow_\beta e_i$ \newline 
This is an example of an expression of the form $e.\ell$, which uses beta evaluation to show that it just returns the expression associated with the label.
So $e.\ell_i$ just returns the expression in $e$ associated with label $\ell_i$ which is $e_i$.
As another example, if we used $\ell_n$ instead of $\ell_i$ then we would just get $e_n$ instead of $e_i$. \newline \newline
$e \equiv_\eta \{ \ell_1 = e.\ell_1; \dots ; \ell_n = e.\ell_n \}$ \newline
This is valid because projecting all of the fields in the record just gives the original record.
You would end up with a record that is identical to the original with $\{ \ell_1 = e_1; \dots \ell_n = e_n \}$ since $e.\ell_i$ gives $e_i$,
meaning for example in the first label $\ell_1 = e.\ell_1$ we just get $\ell_1 = e_1$. Thus they are $\eta$ equivalent. \newline

\section{Record Rules}

\begin{mathpar}
  \infer*[left=Record-I]{\forall i \in \{ 1 , \dots , n \}. \Gamma \vdash e_i : \tau_i}{\Gamma \vdash \{ \ell_1 = e_1; \dots ; \ell_n = e_n \} : \{\ell_1 : \tau_1 ; \dots ; \ell_n : \tau_n \}}
\end{mathpar}

\noindent This rule is for creating a well-typed record.
For any integer $i$ from 1 to $n$, $e_i : \tau_i$ must be well-typed in $\Gamma$.
If it is well-typed for all $i$,
then a record containing all labels and expressions from $\ell_1 = e_1$ to $\ell_n = e_n$, ($\{\ell_1 = e_1 ; \dots ; \ell_n = e_n \}$),
of type $\{\ell_1 : \tau_1 ; \dots ; \ell_n : \tau_n \}$is well-typed in $\Gamma$.

\begin{mathpar}
  \infer*[left=Record-E]{\Gamma \vdash e : \{ \ell_1 : \tau_1 ; \dots ; \ell_n : \tau_n \}}{\Gamma \vdash e.\ell_i : \tau_i}
\end{mathpar}

\noindent If you have a record $e$ that is well-typed in $\Gamma$,
then any arbitrarily chosen label in that record of its matching type,
is well-typed in $\Gamma$. Meaning label $i$ is of type $\tau_i$ in the record and is well-typed in $\Gamma$. \newline

\section{Subtyping}
The relation $\subtype$, which is used like this: $\tau \subtype \tau'$ states that every value of $\tau$ is a value of $\tau'$ and that if someone asks for a $\tau'$,
a value of $\tau$ is valid to be used instead.
However, not every value of $\tau'$ is a value of $\tau$. \newline \newline
This gives us the Liskov Substitution Principle: \newline
If $\tau \subtype \tau'$, then everywhere a $\tau'$ is used, a $\tau$ could just as well be used.
\newline \newline
$\subtype$ is Reflexive and Transitive meaning it is a pre-order on types.

\section{Subtype Rules}

\begin{mathpar}
  \infer*[left=Unit Refl]{\\}{unit \subtype unit}
\end{mathpar}

\noindent This rule is here to show that the subtype relation is reflexive but also that it holds for type $unit$

\begin{mathpar}
  \infer*[left=ProdSub]{\tau_1 \subtype \tau_1' \\ \tau_2 \subtype \tau_2'}{\tau_1 \times \tau_2 \subtype \tau_1' \times \tau_2'}
\end{mathpar}

\noindent This rule shows that subtyping holds for product types.
If $\tau_1 \subtype \tau_1'$ and $\tau_2 \subtype \tau_2'$,
then the product made up of $\tau_1$ and $\tau2$ is a subtype of the product made up of $\tau_1'$ and $\tau_2'$.
This holds as long as the types match up, since $\tau_1$ is on the left side of the product and $\tau_1'$ is also on the left, they match.
However, if the second product was flipped to be $\tau_2' \times \tau_1'$ but the first product stayed the same,
then it would not necessarily hold since $\tau_1$ is not known to be a subtype of $\tau_2'$.

\begin{mathpar}
  \infer*[left=SumSub]{\tau_1 \subtype \tau_1' \\ \tau_2 \subtype \tau_2'}{\tau_1 + \tau_2 \subtype \tau_1' + \tau_2'}
\end{mathpar}

\noindent This rule shows that subtyping holds for sum types. Sum types are like or statements,
so saying $\tau_1 + \tau_2$ is the same as saying either $\tau_1$ or $\tau_2$.
Thinking about the bottom of the rule this way, we can see that it must be true.
Either $\tau_1$ or $\tau_2$ is a subtype of either $\tau_1'$ or $\tau_2'$.
If $\tau_1$ is chosen, then it is a subtype of either $\tau_1'$ or $\tau_2'$,
which is given to be true based on the top of the rule where $\tau_1$ is a subtype of $\tau_1'$.
The same logic works for $\tau_2$ since it is a subtype of $\tau_2'$.

\begin{mathpar}
  \infer*[left=ArrSub]{\tau_1' \subtype \tau_1 \\ \tau_2 \subtype \tau_2'}{\tau_1 \rightarrow \tau_2 \subtype \tau_1' \rightarrow \tau_2'}
\end{mathpar}

\noindent This rule shows that subtyping holds for arrow types ($\tau_1 \rightarrow \tau_2$).
In the top we can see there is contravariance for the first relation since $\tau_1' \subtype \tau_1$ now instead of the other way around like the other rules.
Based on the Liskov Substitution Principle and the Sub rule defined below,
we can replace some of the types in the bottom of the rule to make it more clear.
Since $\tau_1'$ is a subtype of $\tau_1$ that means anywhere where a $\tau_1$ is required,
a $\tau_1'$ can also be used, so now we have $\tau_1' \rightarrow \tau_2 \subtype \tau_1' \rightarrow \tau_2'$.
Now we apply the same logic with $\tau_2 \subtype \tau_2'$ and replace $\tau_2'$ with $\tau_2$ to get $\tau_1' \rightarrow \tau_2 \subtype \tau_1' \rightarrow \tau_2$ which holds by reflexivity.

\begin{mathpar}
  \infer*[left=RecordSub]{\exists f: \{ 1,\dots,m \} \hookrightarrow \{ 1,\dots,n \}.\forall i \in \{ 1,\dots,m \}.\ell_{f(i)} = \ell_i' \wedge \tau_{f(i)} \subtype \tau_i'}{\{ \ell_1 : \tau_1; \dots ; \ell_n : \tau_n \} \subtype \{ \ell_1' : \tau_1' ; \dots ; \ell_m' ; \tau_m' \}}
\end{mathpar}

\noindent This is the last rule for showing that subtyping holds for types.
It shows that subtyping holds for record types. The rule uses an injective function $f$ to match labels.
Since $f$ is an injective function, $m < n$. $\ell_{f(i)} = \ell_i'$ means there is a label that we get using the function $f$ on $i$
which matches up with a label corresponding to $i$ which is at most $m$.
Then we are given that the type given by calling function $f$ on $i$ is a subtype of the type corresponding to $i$ which is at most $m$.
We use this to create the bottom of the rule which says that a record with $n$ labels obtainable using function $f$ is a subtype of the record containing $m$ labels corresponding directly to $i$.
So a record is a subtype of another record if all of the label's types are subtypes of the corresponding label in the supertype record's types,
and at least all of the labels contained in the supertype are in the subtype, but there can be more labels in the subtype than the supertype.

\begin{mathpar}
  \infer*[left=Sub]{\Gamma \vdash e : \tau \\ \tau \subtype \tau'}{\Gamma \vdash e : \tau'}
\end{mathpar}

\noindent This is a simple rule which applies the logic of the Liskov Substitution Principle to apply the supertype to an expression with the type of the subtype.
$e$ is of type $\tau$ and $\tau \subtype \tau'$. This means a $\tau$ can be used in place of anywhere where a $\tau'$ is needed.
So we are just saying that $e$ is of type $\tau'$ to be more specific than saying $e$ is of type $\tau$. \newline
\newline
We can choose to bake the subtyping rules into the typing rules.
It requires us to change the Record-I rule but makes it easier to prove things about subtypes involving substitution.
The new Record-I rule is as follows:

\begin{mathpar}
  \infer*[left=Record-I]{\exists f: \{ 1,\dots,n \} \hookrightarrow \{ 1,\dots,m \}.\forall i \in \{ 1,\dots,m \}.\ell_{f(i)} = \ell_i' \wedge \Gamma \vdash e_{f(i)} : \tau_i}{\Gamma \vdash \{ \ell_1 = e_1; \dots ; \ell_m = e_m \} : \{ \ell_1' : \tau_1 ; \dots ; \ell_n' ; \tau_n \}}
\end{mathpar}

\noindent The rule is similar to RecordSub, but is focused on well-typing of the expressions instead of subtyping.
It still uses the function $f$ to match up labels in the record expression with their labels in the record type. \newline

\section{Extra Info}
$\tau_1 \subtype \tau_2 \wedge \tau_2 \subtype \tau_1 \stackrel{\triangle}{\iff} \tau_1 \equiv \tau_2$ \newline
Fairly straight forward, just states that if two different types are subtypes of each other, then they must be equal,
and the other way around, if two different types are equivalent, then they are subtypes of each other. \newline \newline
$\{ x : \mathbb{N} ; y : \mathbb{N} \} \equiv \{ y : \mathbb{N} : x : \mathbb{N} ; \}$ \newline
This just is just showing that the order that entries of a record are in, does not have an effect on the equivalence of them.

\end{document}

%%% Local Variables:
%%% mode: latex
%%% TeX-master: t
%%% TeX-engine: luatex
%%% TeX-command-default: "Make"
%%% End:
