\documentclass{lecturenotes}

\usepackage[colorlinks,urlcolor=blue]{hyperref}
\usepackage{doi}
\usepackage{xspace}
\usepackage{fontspec}
\usepackage{enumerate}
\usepackage{mathpartir}
\usepackage{pl-syntax/pl-syntax}
\usepackage{forest}
\usepackage{stmaryrd}
\usepackage{epigraph}
\usepackage{xspace}
\usepackage{bbm}
\usepackage{tikz-cd}
\usepackage{unicode-math}

\setsansfont{Fira Code}

\newcommand{\real}{\ensuremath{\mathbb{R}}\xspace}
\newcommand{\complex}{\ensuremath{\mathbb{C}}\xspace}
\newcommand{\abs}[2]{\ensuremath{\lambda #1.\,#2}}
\newcommand{\tabs}[3]{\ensuremath{\lambda #1 \colon #2.\,#3}}
\newcommand{\dbabs}[1]{\ensuremath{\lambda.\,#1}}
\newcommand{\dbind}[1]{\ensuremath{\text{\textasciigrave}#1}}
\newcommand{\app}[2]{\ensuremath{#1\;#2}}
\newcommand{\utype}{\textsf{unit}\xspace}
\newcommand{\unit}{\ensuremath{\textsf{(}\mkern0.5mu\textsf{)}}}
\newcommand{\prodtype}[2]{\ensuremath{#1 \times #2}}
\newcommand{\pair}[2]{\ensuremath{(#1, #2)}}
\newcommand{\projl}[1]{\ensuremath{\pi_1\mkern2mu#1}}
\newcommand{\projr}[1]{\ensuremath{\pi_2\mkern3mu#1}}
\newcommand{\sumtype}[2]{\ensuremath{#1 + #2}}
\newcommand{\injl}[1]{\ensuremath{\textsf{inj}_1\mkern2mu#1}}
\newcommand{\injr}[1]{\ensuremath{\textsf{inj}_2\mkern3mu#1}}
\newcommand{\case}[5]{\ensuremath{\textsf{case}\mkern5mu#1\mkern5mu\textsf{of}\mkern5mu\injl{#2} \Rightarrow #3;\mkern5mu\injr{#4} \Rightarrow #5\mkern5mu\textsf{end}}}
\newcommand{\vtype}{\textsf{void}\xspace}
\newcommand{\vcase}[1]{\ensuremath{\textsf{case}\mkern5mu#1\mkern5mu\textsf{of}\mkern5mu\textsf{end}}}
\newcommand{\rectype}[2]{\ensuremath{\mu #1.\,#2}}
\newcommand{\roll}[1]{\textsf{roll}\mkern2mu#1}
\newcommand{\unroll}[1]{\textsf{unroll}\mkern2mu#1}
\newcommand{\fatype}[2]{\ensuremath{\forall #1.\,#2}}
\newcommand{\Abs}[2]{\Lambda #1.\,#2}
\newcommand{\App}[2]{#1\;[#2]}
\newcommand{\extype}[2]{\ensuremath{\exists #1.\,#2}}
\newcommand{\pack}[3]{\ensuremath{\textsf{pack}\mkern5mu#1\mathrel{\textsf{as}}#2\mathrel{\textsf{in}}#3}}
\newcommand{\unpack}[4]{\ensuremath{\textsf{unpack}\mkern5mu#1\mathrel{\textsf{as}} #2, #3 \mathrel{\textsf{in}} #4}}

\newcommand{\at}{\ensuremath{\mathrel{@}}}
\newcommand{\binval}[4]{\ensuremath{\mathcal{V}(#1 \at #2)(#3 \sim #4)}}
\newcommand{\binexpr}[4]{\ensuremath{\mathcal{E}(#1 \at #2)(#3 \sim #4)}}
\newcommand{\bintypectxt}[2]{\ensuremath{\mathcal{T}(#1)(#2)}}
\newcommand{\binctxt}[4]{\ensuremath{\mathcal{K}(#1 \at #2)(#3 \sim #4)}}

\newcommand{\FV}{\text{FV}}
\newcommand{\BV}{\text{BV}}

\newcommand{\toform}[1]{\ensuremath{\lceil #1 \rceil}}
\newcommand{\totype}[1]{\ensuremath{\lfloor #1 \rfloor}}

\newcommand{\neutral}[1]{#1\;\text{ne}}
\newcommand{\nf}[1]{#1\;\text{nf}}

\newcommand{\subtype}{\ensuremath{\mathrel{\mathord{<}\mathord{:}}}}

\newcommand{\send}[2]{\ensuremath{\textsf{send}\ #1\ \textsf{to}\ #2}}
\newcommand{\recv}[3]{\ensuremath{\textsf{receive}\ #1\ \textsf{from}\ #2\ \textsf{in}\ #3}}
\newcommand{\choosefor}[2]{\ensuremath{\textsf{choose}~#1 \mathrel{\textsf{for}} #2}}
\newcommand{\letchoose}[3]{\ensuremath{\textsf{let}~#1 \mathrel{\textsf{choose}} (\Left \Rightarrow #2; \Right \Rightarrow #3)}}
\newcommand{\senda}[2]{#1 \rightsquigarrow #2}
\newcommand{\recva}[2]{#1 \leftsquigarrow #2}
\newcommand{\choosea}[2]{[#1] \rightsquigarrow #2}
\newcommand{\letchoosea}[2]{[#1] \leftsquigarrow #2}

\title{Asynchrony}
\coursenumber{CSE 410/510}
\coursename{Programming Language Theory}
\lecturenumber{35}
\semester{Spring 2025}
\professor{Professor Andrew K. Hirsch}

\begin{document}
\maketitle

\begin{itemize}
\item Last time we discussed the first \textbf{concurrent $\lambda$~calculus}.
\item During the communcation, sender and receiver can move to the next state
      only when sender is ready to send and receiver is ready to receive.
\item This leads to the \textbf{global synchrony}, However, consider the following program:
      \[
        p \triangleright \send{3}{q}; \recv{x}{q}{x} \parallel q \triangleright \send{4}{p}; \recv{x}{p}{x}
      \]
\item Since both $p$ and $q$ are waiting for each other to receive when they are ready to send,
      this program will never go to the next state, which is a \textbf{deadlock}.
\end{itemize}

\section{Some definitions}\label{sec:asyn-defs}
To avoid confusion between different people doing research in concurrency,
we will use the following definitions for the rest of the lecture:
\begin{itemize}
\item \textbf{Concurrency}: The property of a system or language that allows multiple
      actions to be occur at the same logical time.
\item \textbf{Parallelism}: The property of a system or language that allows multiple
      actions to be occur at the same physical time (or wall-clock time).
\item \textbf{Synchrony}: The property of communication that requires the entire communication
      to be completed before any participant can move to the next state.
\item \textbf{Asynchrony}: The property of communication that allows participants to
      move to the next state before the entire communication is completed.
\end{itemize}
Concurrency and parallelism can exist without each other,
although usually parallelism comes with concurrency.
Also, note that asynchrony and synchrony here are different from
the same terms in the context of distributed systems.

\section{Asynchronous networks} \label{sec:asyn-networks}
\begin{itemize}
\item So we need to add more semantics to the network layer to allow
      the sender and receiver to move to the next state independently.
\item The solution is to use a \textbf{queue} $Q_{pq}$ to describe the messages
      from the sender $p$ to the receiver $q$. Queues act as buffers that allow
      the sender to send messages even if the receiver is not ready to
      receive them. This way, the sender can continue its execution.
\item This can prevent some deadlocks like the one above, but not all, and
      we will talk about that later.
\item We define the syntax of the messages and queues as follows:
      \begin{syntax}
        \category[Messages]{m} \alternative{n} \alternative{[d]}
        \category[Queues]{Q} \alternative{m_1 m_2 \ldots m_n}
      \end{syntax}
\item We use $mQ$ to denote the queue $Q$ with message $m$ at the head of the queue,
      and $Qm$ to denote the queue $Q$ with message $m$ at the tail of the queue.
\item We also define the network as follows:
      \begin{syntax}
            \category[Networks]{\mathcal{N}} \alternative{p_1 \triangleright e_1 \parallel \dots \parallel p_n \triangleright e_n; p_1 p_2 \triangleright Q_{12}; p_1 p_3 \triangleright Q_{13}; \dots; p_{n-1} p_n \triangleright Q_{(n-1)n}}
      \end{syntax}
\item Some notations:
      \begin{itemize}
        \item $\mathcal{N}(p)$: process $p$ is running in the network
        \item $\mathcal{N}[p \mapsto e]$: the network like $\mathcal{N}$ but the process $p$ is running $e$
        \item $\mathcal{N}(pq)$: the queue from $p$ to $q$ in the network
        \item $\mathcal{N}[pq \mapsto Q]$: the network like $\mathcal{N}$ but with the queue from $p$ to $q$ is updated to $Q$
      \end{itemize}
\item Now, we can treat the network is a collection of processes and two directed queues
      between each pair of processes.
      \begin{mathpar}
            \infer{\mathcal{N}(p) \xrightarrow{\varepsilon} e}{\mathcal{N} \to\mathcal{N}[p \mapsto e]} \\
            \infer{\mathcal{N}(p) \xrightarrow{\senda{v}{q}} e\\ \mathcal{N}(pq) = Q}{\mathcal{N} \to \mathcal{N}[p \mapsto e, pq \mapsto Qv]} \and
            \infer{\mathcal{N}(p) \xrightarrow{\choosea{d}{q}} e\\ \mathcal{N}(pq) = Q}{\mathcal{N} \to \mathcal{N}[p \mapsto e, pq \mapsto Q[d]]} \\
            \infer{\mathcal{N}(p) \xrightarrow{\recva{v}{q}} e \\ \mathcal{N}(qp) = vQ}{\mathcal{N} \to \mathcal{N}[p \mapsto e, qp \mapsto Q]}\and
            \infer{\mathcal{N}(p) \xrightarrow{\letchoosea{d}{q}} e\\ \mathcal{N}(qp) = [d]Q}{\mathcal{N} \to \mathcal{N}[p \mapsto e, qp \mapsto Q]}
      \end{mathpar}
\item Here, when $p$ sends out the message, process $q$ is no longer involved in the computation,
      instead we push the message to the end of $Q$.
\item When $p$ receives the message, if there is a message in the queue,
      it must looks like $mQ$.
\end{itemize}

\section{Asynchrony refines synchrony}
\begin{itemize}
\item We can see that the asynchronous network is a refinement of the synchronous network.
\item The synchronous network is just a special case of the asynchronous network
      where the queues are always empty.
\begin{thm}
      If asynchronous network $\mathcal{N}_a$ refines synchronous network $\mathcal{N}_s$,
      $\mathcal{N}_s \rightarrow \mathcal{N}_{s'}$, then $\exists \mathcal{N}_{a'}$ such that
      $\mathcal{N}_{a'}$ refines $\mathcal{N}_{s'}$ with empty queues and $\mathcal{N}_a \rightarrow^\ast \mathcal{N}_{a'}$.
\end{thm}
\begin{proof}
      By case analysis on the evidence for $\mathcal{N}_s \rightarrow \mathcal{N}_{s'}$, there is
      only one interesting case:
      \begin{mathpar}
            \infer
            {\mathcal{N}_s(p) \xrightarrow{\senda{v}{q}} e_1 \\ \mathcal{N}_s(q) \xrightarrow{\recva{v}{q}} e_2}
            {\mathcal{N}_s \to \mathcal{N}_s[p \mapsto e_1, q \mapsto e_2]}
      \end{mathpar}
      In $\mathcal{N}_a$, we can have the following transition:
      \begin{mathpar}
            \infer
            {\mathcal{N}_a(p) \xrightarrow{\senda{v}{q}} e_1 \\ \mathcal{N}_a(pq) = [\cdot]}
            {\mathcal{N}_a \to \mathcal{N}_a[p \mapsto e_1, pq \mapsto v]}
      \end{mathpar}
      But, we have:
      \begin{mathpar}
            \infer
            {(\mathcal{N}_a[p \mapsto e_1, pq \mapsto v])(q) \xrightarrow{\recva{v}{q}} e_2 \\ (\mathcal{N}_a[p \mapsto e_1, pq \mapsto v])(pq) = v}
            {\mathcal{N}_a[p \mapsto e_1, pq \mapsto v] \to \mathcal{N}_a[p \mapsto e_1, q \mapsto e_2, v \mapsto [\cdot]]}
      \end{mathpar}
      We ignore the empty queue, this is the same as $\mathcal{N}_{s'}$.
\end{proof}

\item So, for all the programs that are valid in the synchronous network,
      we can run them in the asynchronous network starting with empty queues,
      but not the other way around (seethe example at the beginning of the lecture).
\item However, this does not guarantee that there will be no deadlocks in the
      asynchronous network. For example, consider the following example:
      \[
        \recv{x}{q}{x} \parallel \recv{y}{p}{y}
      \]
\item In this program, both processes are waiting for each other to send messages,
      and they will never be able to move to the next state even within the asynchronous network.
\end{itemize}

\end{document}
%%% Local Variables:
%%% mode: latex
%%% TeX-master: t
%%% TeX-engine: luatex
%%% End: