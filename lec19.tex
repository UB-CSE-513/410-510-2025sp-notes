\documentclass{lecturenotes}

\usepackage[colorlinks,urlcolor=blue]{hyperref}
\usepackage{doi}
\usepackage{xspace}
\usepackage{fontspec}
\usepackage{enumerate}
\usepackage{mathpartir}
\usepackage{pl-syntax/pl-syntax}
\usepackage{forest}
\usepackage{stmaryrd}
\usepackage{epigraph}
\usepackage{xspace}

\setsansfont{Fira Code}

\newcommand{\abs}[2]{\ensuremath{\lambda #1.\,#2}}
\newcommand{\tabs}[3]{\ensuremath{\lambda #1 \colon #2.\,#3}}
\newcommand{\dbabs}[1]{\ensuremath{\lambda.\,#1}}
\newcommand{\dbind}[1]{\ensuremath{\text{\textasciigrave}#1}}
\newcommand{\app}[2]{\ensuremath{#1\;#2}}
\newcommand{\utype}{\textsf{unit}\xspace}
\newcommand{\unit}{\ensuremath{\textsf{(}\mkern0.5mu\textsf{)}}}
\newcommand{\prodtype}[2]{\ensuremath{#1 \times #2}}
\newcommand{\pair}[2]{\ensuremath{(#1, #2)}}
\newcommand{\projl}[1]{\ensuremath{\pi_1\mkern2mu#1}}
\newcommand{\projr}[1]{\ensuremath{\pi_2\mkern3mu#1}}
\newcommand{\sumtype}[2]{\ensuremath{#1 + #2}}
\newcommand{\injl}[1]{\ensuremath{\textsf{inj}_1\mkern2mu#1}}
\newcommand{\injr}[1]{\ensuremath{\textsf{inj}_2\mkern3mu#1}}
\newcommand{\case}[5]{\ensuremath{\textsf{case}\mkern5mu#1\mkern5mu\textsf{of}\mkern5mu\injl{#2} \Rightarrow #3;\mkern5mu\injr{#4} \Rightarrow #5\mkern5mu\textsf{end}}}
\newcommand{\vtype}{\textsf{void}\xspace}
\newcommand{\vcase}[1]{\ensuremath{\textsf{case}\mkern5mu#1\mkern5mu\textsf{of}\mkern5mu\textsf{end}}}
\newcommand{\rectype}[2]{\ensuremath{\mu #1.\,#2}}
\newcommand{\roll}[1]{\textsf{roll}\mkern2mu#1}
\newcommand{\unroll}[1]{\textsf{unroll}\mkern2mu#1}

\newcommand{\FV}{\text{FV}}
\newcommand{\BV}{\text{BV}}

\newcommand{\toform}[1]{\ensuremath{\lceil #1 \rceil}}
\newcommand{\totype}[1]{\ensuremath{\lfloor #1 \rfloor}}

\newcommand{\neutral}[1]{#1\;\text{ne}}
\newcommand{\nf}[1]{#1\;\text{nf}}

\title{Recursive Types}
\coursenumber{CSE 410/510}
\coursename{Programming Language Theory}
\lecturenumber{19}
\semester{Spring 2025}
\professor{Professor Andrew K. Hirsch}

\begin{document}
\maketitle

\section{Recursive Types}
The types we've discussed so far describe finite sets of values.
However, we often find that we'd like to describe infinite types.
A common example of this is OCaml lists
\[
  \textsf{type 'a list = nil | cons of 'a * 'a list}
\]

Let's see how we'd do this in our current system.
We haven't discussed parameterized types yet, so we'll just rewrite things to work with types we know.
\[
  \textsf{type list = \sumtype{\utype}{\prodtype{\utype}{\textsf{list}}}}
\]

\textsf{type} is an OCaml keyword that lets us write possibly recursive types.
However, the only way we've seen to represent recursion in our $\lambda$-calculi is to use function binding.
\textsf{type} is then just a binder $\mu$.
\[
  \mu \textsf{list. \sumtype{\utype}{\prodtype{\utype}{\textsf{list}}}}
\]

The only difference is that this binder is over types, which we will need to formalise.
We'll do that next.

\section{Intro/Elmination Rules}
We add a few new things to our system.

\begin{syntax}
  \abstractCategory[Type Variables]{X, Y, Z, \dots}
  \category[Types]{\tau} \alternative{\ldots} \alternative{X} \alternative{\rectype{X}{\tau}}
  \category[Expressions]{e} \alternative{\ldots} \alternative{\roll{e}} \alternative{\unroll{e}}
\end{syntax}

\begin{mathpar}
\inferrule*[Left=$\mu$-I]
  {\Gamma \vdash e : \tau[X \mapsto \rectype{X}{\tau}]}
  {\Gamma \vdash \roll{e} : \rectype{X}{\tau}}\and
\inferrule*[Left=$\mu$-E]
  {\Gamma \vdash e : \rectype{X}{\tau}}
  {\Gamma \vdash \unroll e : \tau[X \mapsto \rectype{X}{\tau}]}
\end{mathpar}

We also get $\beta$ and $\eta$ laws
\begin{mathpar}
  \inferrule*[Left=$\cong_\beta$]{ }{\unroll{(\roll e)} \cong_\beta e}\and
  \inferrule*[Left=$\cong_\eta$]{ }{\roll{(\unroll e)} \cong_\eta e}
\end{mathpar}

We note that $\roll{ }$ and $\unroll{ }$ seem to ``Cancel out", in the sense that they bring us from a type $\tau[X \mapsto \rectype{X}{\tau}]$ to $\rectype{X}{\tau}$ and back.
This implies there might be an isomorphism between the two types, which will let us treat them essentially as equal.
All we need to show is that $\unroll{(\roll{e})} = e$ and $\roll{(\unroll{e})} = e$, which is given by our $\beta$ and $\eta$ laws.

\section{Examples}
How can we encode things with these new types?
We need to provide 3 things
\begin{itemize}
  \item A recursive type
  \item Constructors for each branch of the recursive type
  \item Pattern matching for each constructor
\end{itemize}
Here are a few examples
\subsection{Lists}
\subsubsection{Type}
\[
  \textsf{list}\, A \triangleq\, \rectype{X}{\sumtype{\utype}{\prodtype{A}{X}}}\\
\]

\subsubsection{Constructors}
\begin{gather*}
  \textsf{nil} \triangleq\, \roll{(\injl{\unit})}\\
  \textsf{cons} \triangleq\, \tabs{x}{A}{\tabs{xs}{list A}{\roll{(\injr{\pair{x}{xs}})}}}
\end{gather*}

\subsubsection{Pattern Matching}

\begin{gather*}
  \textsf{case of xs nil => $e_1$; y::ys => $e_2$} =\\
  \textsf{\case{(\unroll{xs})}{_}{e_1}{p}{e_2[y \mapsto \projl{p}, ys \mapsto \projr{p}]}}
\end{gather*}

\subsection{Natural Numbers}
\subsubsection{Type}

\[
  \mathbb{N} = \textsf{\rectype{X}{\sumtype{\utype}{X}}}
\]

\subsubsection{Constructors}
\begin{gather*}
  \textsf{0} = \roll{(\injl{\unit})}\\
  \textsf{suc} = \tabs{x}{\mathbb{N}}{\roll{(\injr{x})}}
\end{gather*}

\subsubsection{Pattern Matching}
\begin{gather*}
  \textsf{case n of 0 => $e_1$; suc x => $e_2$} =\\
  \textsf{\case{(\unroll{n})}{_}{e_1}{x}{e_2}}
\end{gather*}

\section{Type Contexts}
A problem still remains with our formulation.
How should we type the following closed program?

\[
  \vdash \tabs{x}{X}{x}
\]

According to our syntax, this is valid, as type variables are now allowed.
However, the meaning of such a function is unknown without knowing what $X$ is.
We therefore have to introduce type contexts.

We will write

\begin{gather*}
  \Delta \vdash \tau\\
  \Delta \vdash \Gamma
\end{gather*}

to mean a type $\tau$ and the types in context $\Gamma$ are well formed.
This means they only reference types in $\Delta$.
We will also add the following to our language.

\begin{syntax}
  \category[Type Context]{\Delta} \alternative{\cdot} \alternative{\Delta, X}
\end{syntax}

And we will add the following rules

\begin{mathpar}
  \infer*[left=Axiom]{X \in \Delta}{\Delta \vdash X}\and
  \infer*[left=Unit]{ }{\Delta \vdash \utype}\\
  \infer*[left=Prod]{\Delta \vdash \tau_1\\ \Delta \vdash \tau_2}{\Delta \vdash \prodtype{\tau_1}{\tau_2}}\and
  \infer*[left=Sum]{\Delta \vdash \tau_1\\ \Delta \vdash \tau_2}{\Delta \vdash \sumtype{\tau_1}{\tau_2}}\and
  \infer*[left=Arrow]{\Delta \vdash \tau_1\\ \Delta \vdash \tau_2}{\Delta \vdash \tau_1 \to \tau_2}\and
  \infer*[left=Rec]{\Delta, X \vdash \tau}{\Delta \vdash \rectype{X}{\tau}}x
\end{mathpar}

We add rules that allow us to create well-formed contexts from other well-formed contexts.
\begin{mathpar}
  \infer*[left=Empty]{ }{\Delta \vdash\cdot} \and
  \infer*[left=Cons]{\Delta \vdash \Gamma\\ \Delta \vdash \tau}{\Delta \vdash \Gamma, x : \tau}
\end{mathpar}

We must change two of our original rules to make sure we're only working with well-formed types
\begin{mathpar}
  \infer*[left=Axiom]{\vdash \Gamma\\ x : \tau \in \Gamma}{\Gamma \vdash x : \tau} \and
  \infer*[left=Unit-I]{\vdash \Gamma}{\Gamma \vdash \unit : \utype}
\end{mathpar}

These rules allow us to write out a new theorem.
\begin{thm}
  If \Gamma\, \vdash\, e : \tau, then \cdot\, \vdash\, \Gamma and \cdot\, \vdash\, \tau
\end{thm}

\section{Untyped $\lambda$-Calculus}
With the power of recursive types, we can do something incredible.
We can actually embed the untyped $\lambda$-calculus within our new typed language.

We do this by defining a recursive type
\[
  D \triangleq \rectype{X}{X \to X}
\]

This allows us to encode terms of untyped $\lambda$-calculus as

\[
  \llbracket e \rrbracket = \begin{cases}
    x & e = x\\
    \roll{(\tabs{x}{D}{\llbracket e \rrbracket})} & e = \lambda. x\, e\\
    \app{(\unroll{e_1})}{e_2} & e = e_1\, e_2
  \end{cases}
\]

Because of the addition of recursive types, we also recover recursive functions from untyped $\lambda$-calculus.
Let us have the following
\[
  \omega_f = \lambda x. f(\app{(\unroll{x})} {x})
\]

Which is typed as

\[
  f : \tau \to \tau \vdash \omega_f : (\rectype{X}{X \to \tau}) \to \tau
\]

This allows us to write
\[
  \textsf{fix} \triangleq \tabs{f}{\tau \to \tau}{\omega_f}: \tau \to \tau
\]

The \textsf{fix} function is what allows us to write loops in untyped $\lambda$-calculus, and it therefore allows us to include recursive functions in our language.
\end{document}
