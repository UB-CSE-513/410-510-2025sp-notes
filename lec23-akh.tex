\documentclass{lecturenotes}

\usepackage[colorlinks,urlcolor=blue]{hyperref}
\usepackage{doi}
\usepackage{xspace}
\usepackage{fontspec}
\usepackage{enumerate}
\usepackage{mathpartir}
\usepackage{pl-syntax/pl-syntax}
\usepackage{forest}
\usepackage{stmaryrd}
\usepackage{epigraph}
\usepackage{xspace}

\setsansfont{Fira Code}

\newcommand{\abs}[2]{\ensuremath{\lambda #1.\,#2}}
\newcommand{\tabs}[3]{\ensuremath{\lambda #1 \colon #2.\,#3}}
\newcommand{\dbabs}[1]{\ensuremath{\lambda.\,#1}}
\newcommand{\dbind}[1]{\ensuremath{\text{\textasciigrave}#1}}
\newcommand{\app}[2]{\ensuremath{#1\;#2}}
\newcommand{\utype}{\textsf{unit}\xspace}
\newcommand{\unit}{\ensuremath{\textsf{(}\mkern0.5mu\textsf{)}}}
\newcommand{\prodtype}[2]{\ensuremath{#1 \times #2}}
\newcommand{\pair}[2]{\ensuremath{(#1, #2)}}
\newcommand{\projl}[1]{\ensuremath{\pi_1\mkern2mu#1}}
\newcommand{\projr}[1]{\ensuremath{\pi_2\mkern3mu#1}}
\newcommand{\sumtype}[2]{\ensuremath{#1 + #2}}
\newcommand{\injl}[1]{\ensuremath{\textsf{inj}_1\mkern2mu#1}}
\newcommand{\injr}[1]{\ensuremath{\textsf{inj}_2\mkern3mu#1}}
\newcommand{\case}[5]{\ensuremath{\textsf{case}\mkern5mu#1\mkern5mu\textsf{of}\mkern5mu\injl{#2} \Rightarrow #3;\mkern5mu\injr{#4} \Rightarrow #5\mkern5mu\textsf{end}}}
\newcommand{\vtype}{\textsf{void}\xspace}
\newcommand{\vcase}[1]{\ensuremath{\textsf{case}\mkern5mu#1\mkern5mu\textsf{of}\mkern5mu\textsf{end}}}
\newcommand{\rectype}[2]{\ensuremath{\mu #1.\,#2}}
\newcommand{\roll}[1]{\textsf{roll}\mkern2mu#1}
\newcommand{\unroll}[1]{\textsf{unroll}\mkern2mu#1}
\newcommand{\fatype}[2]{\ensuremath{\forall #1.\,#2}}
\newcommand{\Abs}[2]{\Lambda #1.\,#2}
\newcommand{\App}[2]{#1\;[#2]}
\newcommand{\extype}[2]{\ensuremath{\exists #1.\,#2}}
\newcommand{\pack}[3]{\ensuremath{\textsf{pack}\mkern5mu#1\mathrel{\textsf{as}}#2\mathrel{\textsf{in}}#3}}
\newcommand{\unpack}[4]{\ensuremath{\textsf{unpack}\mkern5mu#1\mathrel{\textsf{as}} #2, #3 \mathrel{\textsf{in}} #4}}

\newcommand{\FV}{\text{FV}}
\newcommand{\BV}{\text{BV}}

\newcommand{\toform}[1]{\ensuremath{\lceil #1 \rceil}}
\newcommand{\totype}[1]{\ensuremath{\lfloor #1 \rfloor}}

\newcommand{\neutral}[1]{#1\;\text{ne}}
\newcommand{\nf}[1]{#1\;\text{nf}}

\newcommand{\subtype}{\ensuremath{\mathrel{\mathord{<}\mathord{:}}}}

\title{Logical Relations and Normalization}
\coursenumber{CSE 410/510}
\coursename{Programming Language Theory}
\lecturenumber{23}
\semester{Spring 2025}
\professor{Professor Andrew K. Hirsch}

\begin{document}
\maketitle

\begin{itemize}
\item Recall that when we talked about the Propositions as Types Principle, we proved that propositional intuitionistic logic is sound \emph{if STLC + products, sums, unit, and void is normalizing}.
\item However, we didn't have the technology at that time to prove that.
\item Today, we'll start to develop that technology.
\item For all of this week and next, we'll be building up from that technology to do more and more cool things.
\item We're going to start with plain STLC.
\end{itemize}

\section{Why Not Induction?}
\label{sec:why-not-induction}

\begin{itemize}
\item By now you've probably started internalizing a fact about programming languages: most things are proven by induction.
\item So you're probably wondering what goes wrong if you try to prove normalization by induction?
\item Let's imagine such a proof:
\end{itemize}

\begin{thm}[Normalization]
  For every $e$, if $\Gamma \vdash e : \tau$ then there is some $n$ such that $e \to^\ast n$ where $\nf{n}$.
\end{thm}
\noindent\textit{Proof Attempt 1.}\hspace{0.5em}
By induction on the evidence of $\Gamma \vdash e : \tau$.
Most of the cases are easy, except for the following case:

\noindent\textbf{Case $\to$-E:}
We have $e = \app{e_1}{e_2}$ and $\Gamma \vdash e_1 : \tau_1 \to \tau$ and $\Gamma \vdash e_2 : \tau_1$.
$$
\begin{array}{r@{\,\triangleq\,}l}
  \text{IH}_1 & \exists n_1.\; e_1 \to n_1 \mathrel{\text{where}} \nf{n_1}\\
  \text{IH}_2 & \exists n_2.\; e_2 \to n_2 \mathrel{\text{where}} \nf{n_2}
\end{array}
$$
Since $\Gamma \vdash n_1 : \tau_1$, we know that either $n_1$ is neutral, or $n_1 = \tabs{x}{\tau_1}{n'_1}$ for some $x$ and $n'_1$, where $n'_1$ is also in normal form.
Further, we know that
$$e = \app{e_1}{e_2} \to^\ast_\beta \app{n_1}{n_2}$$
If $n_1$ is neutral, then $\app{n_1}{n_2}$ is neutral and therefore normal, so we are done.
Otherwise, we get
$$\app{n_1}{n_2} = \app{(\tabs{x}{\tau_1}{n'_1})}{n_2} \to n'_1[x \mapsto n_2]$$
However, now we have a problem: our IH doesn't apply.
In fact, there's not really a way to get an IH that applies using the ``normal'' inductive methods.
So we abort this proof.
\begin{flushright}
  $\lightning$
\end{flushright}

\section{A Logical Relation}
\label{sec:logical-relation}

\begin{itemize}
\item So how do we fix this?
\item Intuitively, the idea here is that a type $\tau$ \emph{means} something: it tells us something about how a program with that type behaves.
\item Every type $\tau$ means that a program with that type terminates.
\item But more importantly, the type $\tau_1 \to \tau_2$ tells us that any term $f$ with this type can take a term $e$ with the type $\tau_1$ and produce something of type $\tau_2$.
\item So, what if instead of defining typing \emph{syntactically} as an inductive relation on terms, we define the meaning of a type as a constraint on behavior?
  Then, any term that behaves according to that constraint can be said to be \emph{semantically} in that type (even if it's not well typed by the syntactic type system).
\item Clearly, we want our syntactic type system to enforce this semantic behavior, but it may be overly conservative.
\item Why didn't we just do this to begin with?
  Well, the idea of ``a program behaves according to some constraint'' is super undecidable.
  But we want typing to be decidable.
  The syntactic relation we gave earlier is easy to decide!
\item Now, however, we want to \emph{prove} something about typing.
  What we care about is that constraint, and we don't have to figure out whether any particular program has it.
  Therefore, doing this more-semantic thing is worthwhile.
\item So, our job is to build a predicate $\mathcal{P} \colon \textsf{type} \to \textsf{term} \to \textsf{Set}$.
  The idea is to do so by recursion on the first argument.
\item  We write $\mathcal{P}(\tau)(e)$ to mean ``$e$ semantically behaves according to the constraints of type $\tau$.''
\item We define $\mathcal{P}$ as follows:
\end{itemize}

$$
\begin{array}{r@{\;\triangleq\;}l}
  \mathcal{P}(A)(e) & \exists n.\; \nf{n} \land e \to^\ast n\\
  \mathcal{P}(\tau_1 \to \tau_2)(e) & \exists n.\;\nf{n} \land e \to^\ast n \land (\forall e'.\; \mathcal{P}(\tau_1)(e') \to \mathcal{P}(\tau_2)(\app{e}{e'}))
\end{array}
$$

\begin{itemize}
\item Notice two things about this:
  \begin{enumerate}
  \item Clearly, if $\mathcal{P}(\tau)(e)$, then $e$ steps to a normal form $n$.
  \item If $\mathcal{P}(\tau_1 \to \tau_2)(e)$, then we can apply it to $\tau_1$~terms to get a $\tau_2$~term.
  \end{enumerate}
\item This tells us that we satisfied three principles of building logical relations:
  \begin{enumerate}
  \item The relation should only contain terms of interest.
    (For us, all terms are of interest and so this is trivial, but sometimes you're only interested in particular terms.)
  \item The property of interest should be baked into the relation.
    This is point (1) above.
  \item The property of interest should be preserved by all elimination forms.
    This is point (2) above.
  \end{enumerate}
\item It may seem like this won't obviously fix the problem we had before: how do we say anything about substitution here?
\item Well, we will generalize our theorem to include substitutions, allowing ourselves to have a strong induction hypothesis.
\item In order to do this, we define another relation, this time of type $\mathcal{C} : \textsf{Context} \to \textsf{Substitution} \to \textsf{Set}$:
  $$
  \mathcal{C}(\Gamma)(\gamma) \triangleq \forall x : \tau \in \Gamma.\; \mathcal{P}(\tau)(\gamma(x))
  $$
  Intuitively, this means that $\gamma$ will replace everything in $\Gamma$ with something that behaves like the appropriate type.
  It is traditional to use $\gamma$ (lower-case gamma) for substitutions in this setting.
\item We can then say the following:
  $$\Gamma \vDash e : \tau \triangleq \forall \gamma.\; \mathcal{C}(\Gamma)(\gamma) \to \mathcal{P}(\tau)(e[\gamma])$$
  In other words, for any way to fill in the free variables with terms that behave correctly, $e$ behaves correctly according to $\tau$.
  We say ``$e$ is semantically well typed at $\tau$ in the context $\Gamma$.''
\item Note that for any $\tau$, $\mathcal{P}(\tau)(x)$ holds.
\item We can then prove the following theorem:
\end{itemize}


\begin{thm}[Fundamental Theorem]
  If $\Gamma \vdash e : \tau$, then $\Gamma \vDash e : \tau$.
\end{thm}
\begin{proof}
  By induction on the evidence that $\Gamma \vdash e : \tau$.

  \noindent\textbf{Case \textsc{var}:}
  In this case, $e = x$.
  Let $\gamma$ be such that $\mathcal{C}(\Gamma)(\gamma)$; then $\gamma(x)$ is defined and $\mathcal{P}(\tau)(\gamma(x))$, since $x : \tau \in \Gamma$.
  But $e[\gamma] = \gamma(x)$, so we're done.

  \noindent\textbf{Case \textsc{$\to$-I}:}
  In this case, $e = \tabs{x}{\tau_1}{e'}$, $\tau = \tau_1 \to \tau_2$, and $\Gamma, x : \tau_1 \vdash e' : \tau_2$.
  We need to show that there is an $n$ such that $e[\gamma] \to^\ast n$, and that for any $e''$ such that $\mathcal{P}(\tau_1)(e'')$, we have $\mathcal{P}(\tau_2)(\app{e}{e''})$.
  By IH, we know that $\Gamma, x : \tau_1 \vDash e' : \tau_2$.

  Let $\gamma$ be such that $\mathcal{C}(\Gamma)(\gamma)$.
  Then $\mathcal{C}(\Gamma, x : \tau_1)(\gamma[x \mapsto x])$, and so by IH $\mathcal{P}(\tau_2)(e'[\gamma[x \mapsto x]]) = \mathcal{P}(\tau_2)(e'[\gamma])$.
  Thus, there is a normal-form~$n'$ such that $e' \to^\ast n'$.
  Thus, $e = \tabs{x}{\tau_1}{e'} \to^\ast \tabs{x}{\tau_1}{n'}$.

  Finally, let $e''$ be such that $\mathcal{P}(\tau_1)(e'')$.
  Then $\mathcal{C}(\Gamma, x : \tau)(\gamma[x \mapsto e''])$, and so $\mathcal{P}(\tau_2)(e'[\gamma[x \mapsto e'']])$.
  But we have $$\app{(e[\gamma])}{e''} = \app{(\tabs{x}{\tau_1}{e'[\gamma]})}{e''} \to e'[\gamma][x \mapsto e] = e'[\gamma[x \mapsto e]]$$
  and so we're done.

  \noindent\textbf{Case \textsc{$\to$-E}:}
  In this case, we have $e = \app{e_1}{e_2}$ where $\Gamma \vdash e_1 : \tau_1 \to \tau$ and $\Gamma \vdash e_2 : \tau_1$.
  Moreover, by IH we have $\Gamma \vDash e_1 : \tau_1 \to \tau$ and $\Gamma \vDash e_2 : \tau_1$.

  Let $\gamma$ be such that $\mathcal{C}(\Gamma)(\gamma)$.
  Then $e[\gamma] = (\app{e_1}{e_2})[\gamma] = \app{(e_1[\gamma])}{(e_2[\gamma])}$.
  But by IH, we know that $\mathcal{P}(\tau_1 \to \tau)(e_1[\gamma])$ and $\mathcal{P}(\tau_1)(e_2[\gamma])$.
  The result now comes directly from the definition of $\mathcal{P}(\tau_1 \to \tau)$.  
\end{proof}

\begin{cor}[Normalization]
  If $\Gamma \vdash e : \tau$, then there is a normal-form~$n$ such that $e \to^\ast n$.
\end{cor}
\begin{proof}
  Since $\Gamma \vdash e : \tau$, we know that $\Gamma \vDash e : \tau$ by the fundamental theorem.
  Note that the identity substitution ($\text{id}_\Gamma \triangleq [x \mapsto x \mid x \in \text{dom}(\Gamma)]$) is a good substitution: $\mathcal{C}(\Gamma)(\text{id}_\Gamma)$.
  Moreover, $e[\text{id}_\Gamma] = e$.
  Thus, $\mathcal{P}(\tau)(e[\text{id}_\Gamma]) = \mathcal{P}(\tau)(e)$.
  The result immediately follows.
\end{proof}

\begin{itemize}
\item In order to get our desired result, we need to extend our logical relation to cover our entire type system.
\item We can do that as follows:
\end{itemize}

$$
\begin{array}{r@{\;\triangleq\;}l}
  \mathcal{P}(\utype)(e) & \exists n.\,\nf{n} \land e \to^\ast n \land (\neutral{n} \lor n = \unit)\\
  \mathcal{P}(\vtype)(e) & \exists n.\,\neutral{n} \land e \to^\ast n\\
  \mathcal{P}(\prodtype{\tau_1}{\tau_2})(e) & (\exists n.\, \nf{n} \land e \to^\ast n) \land \mathcal{P}(\tau_1)(\projl{e}) \land \mathcal{P}(\tau_2)(\projr{e})\\
  \mathcal{P}(\sumtype{\tau_1}{\tau_2})(e) & (\exists n.\,\nf{n} \land e \to^\ast n) \land \forall \tau, e_1, e_2.\,\begin{array}[t]{l}(\forall v.\, \mathcal{P}(\tau_1)(v) \to \mathcal{P}(\tau)(e_1[x \mapsto v])) \to\\
    (\forall v.\, \mathcal{P}(\tau_2)(v) \to \mathcal{P}(\tau)(e_2[y \mapsto v])) \to\\
    \mathcal{P}(\tau)(\case{e}{x}{e_1}{y}{e_2})
    \end{array}
\end{array}
$$

\begin{cor}[Consistency]
  $\not\vdash \bot$
\end{cor}
\begin{proof}
  Recall that any proof of $\vdash \bot$ corresponds to a typing derivation of $\vdash e : \vtype$ for some $e$.
  But then, by normalization, we know that $e$ normalizes, so there is some normal-form~$n$ such that $e \to^\ast n$.
  By preservation, $\vdash n : \vtype$.
  But there is no such thing as a normal-form term of type $\vtype$!
  Thus, there must not be any such proof of $\vdash \bot$.
\end{proof}

\end{document}

%%% Local Variables:
%%% mode: latex
%%% TeX-master: t
%%% TeX-engine: luatex
%%% End:
